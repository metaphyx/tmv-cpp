% !TEX root = TMV_Documentation.tex

\section{Known bugs and deficiencies}
\label{Bugs}
\index{Bugs!Known bugs}

\subsection{Known compiler issues}
\index{Installation!Known problems}
\index{Bugs!Known compiler issues}
\label{Install_Issues}

I have tested the code using the following compilers:\\
$\quad$\\
% Done:
% Flute: clang++ 3.1 NO BLAS
% Flute: Apple g++ 4.0.1 NO BLAS
% Flute: Apple g++ 4.2.1 NO BLAS
% Flute: g++ 4.4.6 NO BLAS
% Flute: g++ 4.5.3 NO BLAS, ATLAS, ATLAS_LAP
% Flute: g++ 4.6.2 NO BLAS, GOTO2, GOTO2+FLAPACK, ATLAS
% bnl: g++ 4.3.2 NO BLAS, FBLAS
% bnl: g++ 4.1.2 NO BLAS, FBLAS
% apu: icpc 10.1 NO BLAS, MKL BLAS, MKL LAPACK
% chimaera: pgCC 6.1 NO BLAS, ACML BLAS, ACML LAPACK
% chimaera: g++ 3.4.6 NO BLAS, ATLAS+FLAPACK, 
% lonestar: icpc 11.1 NO BLAS, MKL BLAS, MKL LAPACK
% lonestar: g++ 4.1.2 NO BLAS, FBLAS, MKL BLAS, MKL LAPACK
% queenbee: icpc 11.1 NO BLAS, MKL BLAS, MKL LAPACK
% queenbee: g++ 3.4.6 NO BLAS, FBLAS, MKL BLAS, MKL LAPACK
% susi: g++ 4.4.5 NO BLAS, MKL BLAS, MKL LAPACK	
% susi: icpc 12.0.4 NO BLAS, MKL BLAS, MKL LAPACK
% 
GNU's g++ -- versions 3.4.6, 4.1.2, 4.3.2, 4.4.7, 4.5.3, 4.6.2, 4.8.2 \\
% chimaera, bach g++41, bach g++, bnl g++, Flute g++-fsf-4.5, Flute g++-fsf-4.6
Apple's g++ -- version 4.0.1, 4.2.1 \\
% Flute g++-4.0, g++-4.2
Intel's icpc -- versions 10.1, 11.1, 12.0.4, 14.0.1\\
% apu, queenbee, susi, bnl
Portland's pgCC -- version 6.1\\
% chimaera
LLVM's clang++ -- version 3.1\\
% Flute
Apple's clang++ (erroneously named g++) -- versions 425, 500\\
% Cello, Flute
Microsoft's cl -- Visual C++ 2008 Express Edition\\
% Sousaphone
\index{Installation!g++}
\index{Installation!icpc}
\index{Installation!pgCC}
\index{Installation!clang++}
\index{Installation!Visual C++}

It should work with any ansi-compliant
compiler, but no guarantees if you use one other than these\footnote{
It does seem to be the case that 
every time I try the code on a new compiler, there is some issue that needs to be addressed.  
Either because the compiler fails to support some aspect of the C++ standard, or they enforce
an aspect that I have failed to strictly conform to.  Or sometimes even because of a bug in the compiler.}.
  So if you do try to compile on a different compiler, 
I would appreciate it if you could let me know whether you were successful.  
Please email the TMV discussion group at \mygroup\ with your experience (good or bad) using 
compilers other than these.

There are a few issues that I have discovered when compiling with various 
versions of compilers, and I have usually come up with a work-around for
the problems.  So if you have a problem, check this list to see if a solution
is given for you.  

\begin{itemize}
\item {\bf g++ versions 4.4 and 4.5 on Snow Leopard:}
\index{Installation!g++}
\index{Bugs!Exceptions not handles correctly when using g++ version 4.4 or 4.5 on Snow Leopard}
g++ 4.4 and 4.5 (at least 4.4.2 through 4.5.0) have a pretty egregious bug in their exception handling that 
can cause the code to abort rather than allow a thrown value to be caught.  
This is a reported bug (42159), and according to the bug report it seems to only occur in
conjunction with MacOS 10.6 (Snow Leopard), however it is certainly possible that it 
may show up in other systems too.  (I don't have Lion yet, so I haven't tested it with that.)

However, I have discovered a workaround that seems to fix the problem.  Link with 
\tt{-lpthread} even if you are not using OpenMP or have any other reason to use that
library.  Something about linking with that library fixes the exception handling problems.
Anyway, this is mostly to explain why TMV adds \tt{-lpthread} to the recommended linking instructions for these systems even if you disable OpenMP support.

\item {\bf g++ -O2, versions 4.1 and 4.2:}
\index{Installation!g++}
\index{Bugs!Wrong answers when using g++ version 4.1 or 4.2}
It seems that there is some problem with the -O2 optimization of g++ versions 4.1.2 and 4.2.2
when used with TMV debugging turned on.  Everything compiles fine when I use
\texttt{g++ -O} or \texttt{g++ -O2 -DNDEBUG}.  But when I compile with \texttt{g++ -O2} (or \texttt{-O3}) without
\texttt{-DNDEBUG}, then the test suite fails, getting weird results for some arithmetic operations
that look like uninitialized memory was used.  

I distilled the code down to a small code snippet that still failed 
and sent it to Gnu as a bug report.
They confirmed the bug and suggested
using the flag \texttt{-fno-strict-aliasing}, which did fix the problems.

Another option, which might be a good idea anyway is to just use \texttt{-O1} 
when you want a version that 
includes the TMV assert statements, and make sure to use \texttt{-DNDEBUG} 
when you want a more optimized version.

\item {\bf Apple g++:}
\index{Installation!g++}
Older versions of Apple's version of g++ that they shipped with the Tiger OS did not work for 
compilation of the TMV library.  It was called version 4.0, but I do not remember the build number.
They seem to have fixed the problem with later XCode update,
but if you have an older Mac and want to compile TMV on it and the native g++ 
is giving you trouble,
you should either upgrade to a newer Xcode distribution or download the real GNU gcc instead;  
I recommend using Fink (\url{http://fink.sourceforge.net/}).

\item {\bf pgCC:}
\index{Installation!pgCC}
I only have access to pgCC version 6.1, so these notes only refer to that version.
\begin{itemize}
\item
Apparently pgCC does not by default support exceptions when compiled 
with openmp turned on.  
So if you want to use the parallel versions of the algorithms,
you need to compile with the flags \texttt{-mp --exceptions}.  This is a documented feature,
but it's not very obvious, so I figured I'd point it out here.

\item 
There was a bug in pgCC version 6.1 that was apparently fixed in version 7.0 where
\tt{long double} variables were not correctly written with \tt{std::ostream}.  The values
were written as either \tt{0} or \tt{-}.  So I have written a workaround in the code for
pgCC versions before version 7.0\footnote{
Thanks to Dan Bonachea for making available his file \texttt{portable\_platform.h},
which makes it particularly easy to test for particular compiler versions.},
where the \tt{long double} values are copied to 
\tt{double} before writing.  This works, but only for values that are so convertible.
If you have values that are outside the range representable by a \tt{double}, then 
you may experience overflow or underflow on output.

\end{itemize}

\item {\bf Borland's C++ Builder:}
\index{Installation!Borland C++ Builder}
I tried to compile the library with Borland's C++ Builder for Microsoft Windows
Version 10.0.2288.42451 Update 2, but it failed at fairly foundational aspects of the 
code, so I do not think it is possible to get the code to work.  However, if somebody wants
to try to get the code running with this compiler or some other Borland product, 
I welcome the effort and would
love to hear about a successful compilation (at \mygroup).

\item {\bf Solaris Studio:}
\index{Installation!Solaris Studio}
I also tried to compile the library with Sun's Solaris Studio 12.2, which includes
version 5.11 of CC, its C++ compiler.
I managed to get the main library to compile, but the test suite wouldn't compile.
The compiler crashed with a ``Signal 11'' error.  
(Actually tmvtest1 and tmvtest3a, 3d and 3e worked.  But not the others.)
I messed around with it for a 
while, but couldn't figure out a workaround.  However, it may be the case that 
there is something wrong with my installation of CC, since it's not something
I really use, and I installed it on a non-Sun machine, so who knows how reliable 
that is.  Since I couldn't get it working completely,
I'm not willing to include this on my above list of ``supported'' compilers,
but if you use this compiler regularly and want to vet the TMV code for me, I would 
appreciate hearing about your efforts at  \mygroup.

\item {\bf Memory requirements:}
The library is pretty big, so it can take quite a lot of memory to compile. 
For most compilers, it seems that a minimum of around 512K is required.
For compiling the test suite with the \texttt{XTEST} flag set to something other than 0 (especially something that combines a lot of multiple extra tests, e.g. XTEST=127), at least around 4 GB of memory is recommended.

\item {\bf Linker choking:}
Some linkers (e.g. the one on my old Mac G5) have trouble with the size 
of some of the test suite's executables, especially when compiled 
with \texttt{XTEST} set to include lots of extra tests.  If you encounter this problem, you can instead
compile the smaller test suites.  

The tests
in \texttt{tmvtest1} are split into \texttt{tmvtest1a}, \texttt{tmvtest1b} and \texttt{tmvtest1c}.
Likewise \texttt{tmvtest2} has \texttt{a}, \texttt{b} and \texttt{c} versions, and \texttt{tmvtest3}
has \texttt{a}, \texttt{b}, \texttt{c} and \texttt{d} versions.  These are 
compiled by typing \texttt{scons test SMALL\_TESTS=true}.  Or you can make them one
at a time by typing \texttt{scons test1a},~ \texttt{scons test1b}, etc.

You might also try testing only one type at a time: First compile with \texttt{INST\_FLOAT=false}
and then \texttt{INST\_FLOAT=true INST\_DOUBLE=false}.  This cuts the size of the executables
in half, which can also help if the above trick doesn't work.  (I had to do this for test2c on one 
of my test systems that does not have much memory.)
\index{Installation!Test suite}

\item {\bf Non-portable IsNaN():}
The LAPACK function \tt{?stegr} sometimes produces \tt{nan} values on output.
Apparently there is a bug in at least some distributions of this function.
Anyway, TMV checks for this and calls the alternative (but slower) function
\tt{?stedc} instead whenever a \tt{nan} is found.  
The problem is that there is no C++ standard way to check
for a \tt{nan} value.  

The usual way is to use a macro \tt{isnan}, which is usually
defined in the file \tt{<math.h>}.  However, this is technically a C99 extension,
not standard C++.  So if this macro is not defined, then TMV tries two other
tests that usually detect \tt{nan} correctly.  But if this doesn't work correctly
for you, then you may need to edit the file \texttt{src/TMV\_IsNaN.cpp} to work
with your system\footnote{
However, the provided code did work successfully on all the compilers I 
tested it on, so technically this is not a ``known'' compiler issue, just a 
potential issue.}.

Note that this can only ever be an issue if you specifically request that TMV use the \tt{stegr} algorithm with the SCons option \tt{USE_STEGR=true}.  If you do not do this, then TMV doesn't use the LAPACK \tt{?stegr} functions at all, so it's not a problem.

\end{itemize}

\subsection{Known problems with BLAS or LAPACK libraries:}
There are a number of possible errors that are related to particular BLAS or LAPACK
distributions, or combinations thereof:
\begin{itemize}

\item{\bf Strict QRP decomposition fails:}
\index{LAPACK!Problems with QRP decomposition}
Some versions of the LAPACK function \tt{dgeqp3} do not produce the correct
result for $R$.  The diagonal elements of $R$ are supposed to be monotonically decreasing
along the diagonal, but sometimes this is not strictly true.  This probably varies among
implementations, so your version might always succeed.

This would only happen if you install TMV with the SCons option \tt{USE_GEQP3=true},
to use the LAPACK function (which is called \tt{?geqp3} rather than the native TMV function.  So the solution is to use \tt{USE_GEQP3=false} instead.

\item{\bf Info $<$ 0 errors:}
\index{LAPACK!Workspace issues}
If you get an error that looks something like:\\
\texttt{TMV Error: info < 0 returned by LAPACK function dormqr}\\
then this probably means that your LAPACK distribution does not support
workspace size queries.  The solution is to use the flag \texttt{-DNOWORKQUERY}.
I think TMV usually knows about these cases now and checks for them, but you might have a LAPACK distribution that I'm not aware of.  So you can add this flag by hand with the SCons option \tt{EXTRA_FLAGS=-DNOWORKQUERY}.  If that doesn't fix your problem, please file a bug report at  \myissues, since that means it's probably something else.

\item {\bf Unable to link LAPACK on Mac:}
\index{LAPACK!Linking}
\index{LAPACK!CLAPACK}
I get a linking error on my Mac when combining the XCode LAPACK library
(either libclapack.dylib or liblapack.dylib) with a non-XCode BLAS library.
Specifically, it can't find the \texttt{?getri} functions.   Basically, the reason is that
the XCode LAPACK libraries are designed to be used with one of the BLAS
libraries, libcblas.dylib or libblas.dylib, in the same directory, 
and that BLAS library has the \texttt{getri}
functions, even though they are properly LAPACK functions.
Anyway, it's basically a bug in the Apple XCode distribution of these files,
and the result is that if you use a different BLAS library, the linking fails.

The solution is either to use the XCode BLAS library or to install your own CLAPACK
library.  If you do the latter, you will probably want to rename (or delete) these Mac
library files, since they are in the /usr/lib directory, and \tt{-L} flags usually can't take 
precedence over /usr/lib.

\item {\bf Errors in \tt{Matrix<float>} calculations using BLAS on Mac:}
\index{BLAS!Errors with float}
\index{Bugs!Wrong answers in \tt{Matrix<float>} when using XCode BLAS on Mac}
I have found that some XCode BLAS libraries seem to have errors in the 
calculations for large \tt{<float>} matrices.  I think it is an error in the \tt{sgemm}
function.  Since very many of the other algorithms use this function, the error
propagates to lots of other routines as well.  The errors seem to be only for
large matrices with at least one dimension $N > 100$ or so.  (I'm not sure of the 
exact crossover point from working to non-working code.)

Apparently, this is a known bug (ID 7437011), but I don't know if it is fixed in the latest
XCode 4 for the Lion OS (I don't yet have a computer with Lion on it).  But the bug is
at least present in XCode versions 3.2.2 through 3.2.6.
The best thing to do to see if your XCode installation has this problem
is to install the test suite and make sure that it runs without errors.  If it makes it 
through without errors, then you don't have to worry about it.

If you encounter this problem, which manifests as \tt{tmvtest1} failing in the \tt{<float>} section,
then I'm afraid the possible solutions are to forego using \tt{Matrix<float>}, switch to another
BLAS library, or compile TMV without BLAS (with SCons option \tt{USE_BLAS=false}).

\item {\bf Overflow and underflow when using LAPACK:}
\index{LAPACK!Overflow/underflow problems}
\index{Bugs!nan or inf when using LAPACK}
The normal distribution of LAPACK algorithms are not as careful as the TMV code when 
it comes to guarding against overflow and underflow.  As a result, you may find
matrix decompositions resulting in values with \tt{nan} or \tt{inf} when using TMV
with LAPACK support enabled.  The solution if this is a problem for you is to compile TMV without the LAPACK
library (using the SCons option \tt{WITH_LAPACK=false}).
This does not generally result in slower code, since the native TMV code is almost always
as fast (usually using the same algorithm) as the LAPACK distribution, but has better
guards against overflow and underflow.

\item {\bf Segmentation fault or other errors when using LAPACK in multi-threaded program:}
\index{LAPACK!Problems in multi-threaded programs}
I have found that many LAPACK libraries (especially older installations) 
are not thread-safe.  So you might get segmentation faults or other strange
non-repeatable errors at random times when using LAPACK within a multi-threaded
program.  The solution is to compile TMV without LAPACK (with SCons option
\tt{WITH_LAPACK=false}).

\item {\bf LAPACK fails for extremely large matrices:}
\index{LAPACK!Problems with extremely large matrices}
For matrices that use more than 2 GBytes of memory, the integers that index the memory addresses need to be more that $2^{31}$.  This number exceeds the capacity of a 32 bit (signed) integer.  So the TMV library uses \tt{ptrdiff_t} for all integers that might be used in calculating such an index to make sure it is always safe.  However, LAPACK distributions generally use \tt{int}.  If this happens to be 32 bits on your machine, it can overflow.  This is particularly a problem for the LAPACK SVD routine, since it needs workspace of size a bit more than $4n^2$.  So if $4n^2$ is more than $2^{31}$, it will fail.  Again, the solution is to compile TMV without LAPACK (with SCons option \tt{WITH_LAPACK=false}).

\end{itemize}

\subsection{To-do list}
\label{To_Do_List}

Here is a list of various deficiencies with the current version of the TMV library.
These are mostly features that are not yet included, rather than bugs per se.

If you find something to add to this list, or if you want me to bump something
to the top of the list, let me know.  Not that the list is currently in any kind of 
priority order, but you know what I mean.  Please post a feature request
at \myissues, or email the discussion group about what you need at \mygroup.

\begin{enumerate}
\item
\textbf{Symmetric arithmetic}
\index{SymMatrix!Arithmetic}
\index{Bugs!SymMatrix arithmetic}

When writing complicated equations involving complex symmetric or hermitian matrices, 
you may find that an equation that seems perfectly ok does not compile.
The reason for this problem is explained in \S\ref{SymMatrix_Arithmetic} in some detail, 
so you should read about it there.  But basically, the workaround is usually
to break your equation up into smaller steps that do not require the code to 
explicitly instantiate any matrices.  For example: (this is the example from \S\ref{SymMatrix_Arithmetic})
\begin{tmvcode}
s3 += x*s1 + s2;
\end{tmvcode}
will not compile if \tt{s1}, \tt{s2}, and \tt{s3} are all complex symmetric, even though it is 
valid, mathematically.  Rewriting this as:
\begin{tmvcode}
s3 += x*s1;
s3 += s2;
\end{tmvcode}
will compile and work correctly.  This bug will be fixed in version 0.90, which is currently
in development.

\item
\textbf{Eigenvalues and eigenvectors}
\index{Eigenvalues}

The code only finds eigenvalues and eigenvectors for hermitian matrices.
I need to add the non-hermitian routines.

\item
\textbf{More special matrix varieties}

Block-diagonal, generic sparse (probably both
row-based and column-based), block sparse, symmetric and hermitian block
diagonal, etc.  Maybe skew-symmetric and skew-hermitian.  Are these worth adding?  Let me know.

\item
\textbf{Packed storage}
\index{SymMatrix!Packed storage}
\index{UpperTriMatrix!Packed storage}
\index{LowerTriMatrix!Packed storage}

Triangular and symmetric matrices. can be stored in (approximately) half the 
memory as a full $N \times N$ matrix using what is known as packed storage.  
There are BLAS routines for dealing with
these packed storage matrices, but I don't yet have the ability to 
create/use such matrices.

\item
\textbf{Hermitian eigenvector algorithm}
\index{SymMatrix!Eigenvalues and eigenvectors}
\index{SymBandMatrix!Eigenvalues and eigenvectors}
\label{Bugs_RRR}

There is a faster algorithm for calculating eigenvectors of a hermitian
matrix given the eigenvalues, which uses a technique know as
a ``Relatively Robust Representation''.  The native TMV
code does not use this, so it is slower than a compilation which calls
the LAPACK routine (which is only enabled if you use both 
\tt{WITH_LAPACK=true} and \tt{USE_STEGR=true}).
I think this is the only routine for which the LAPACK version is still significantly
faster than the native TMV code.  Although, I do find that many LAPACK distributions have a fairly buggy version of the algorithm that does not deal very well with overflow and underflow, which is why it must be specifically enabled.  Hopefully when I implement this in TMV, I can do a better job in this regard.

\item
\textbf{Row-major Bunch-Kaufman}
\index{SymMatrix!Bunch-Kaufman decomposition}
\index{Bunch-Kaufman decomposition!SymMatrix}

The Bunch-Kaufman decomposition for row-major symmetric/hermitian
matrices is currently $L D L^\dagger$, rather than $L^\dagger D L$.  
The latter should be somewhat (30\%?) faster.  The current $L D L^\dagger$
algorithm is the faster algorithm for column-major matrices.\footnote{
These comments hold when the storage of the symmetric matrix is in the 
lower triangle - it is the opposite for storage in the upper triangle.}

\item
\textbf{Conditions}

Currently, SVD is the only decomposition that calculates the condition
of a matrix (specifically, the 2-condition).  
LAPACK has routines to calculate the 1- and infinity-condition
from an LU decomposition (and others).  I should add a similar capability.

\item
\textbf{Division error estimates}

LAPACK provides these.  It would be nice to add something along the same lines.

\item
\textbf{Equilibrate matrices}

LAPACK can equilibrate matrices before division.  Again, I should include this
feature too.  Probably as an option (since most matrices don't need it)
as something like \tt{m.Equilibrate()} before calling a division routine.

\item
\textbf{OpenMP in the SVD algorithm}
\index{OpenMP}

TMV's implementation of the divide-and-conquer SVD algorithm only uses half of the potential for parallelization at the moment.  I need to reorganize the algorithm a bit to make it more amenable to being further parallelized, but it is certainly doable.

\item
\textbf{Check for memory throws}

Many algorithms are able to increase their speed by allocating extra
workspace.  Usually this workspace is significantly smaller than the
matrix being worked on, so we assume there is enough space for 
these allocations.  However, I should add try-catch blocks to catch 
any out-of-memory throws and use a less memory-intesive algorithm
when necessary.

Version 0.90 will include these checks.

\item
\textbf{Integer determinants for special matrices}

The determinant of \tt{<int>} matrices is only written for a regular \tt{Matrix},
and of course for the trivial \tt{DiagMatrix} and \tt{TriMatrix} types.  But
for \tt{BandMatrix}, \tt{SymMatrix}, and \tt{SymBandMatrix}, I just copy to a regular
\tt{Matrix} and then calculate the determinant of that.  I can speed up the 
calculation for these special matrix types by taking advantage of their special
structure, even using the same Bareiss algorithm as I currently use.
\index{BandMatrix!Determinant!integer values}
\index{SymMatrix!Determinant!integer values}
\index{SymBandMatrix!Determinant!integer values}

\end{enumerate}

\subsection{Reporting bugs}

If you find a bug in the TMV library that is not listed above, please let me know.  The preferred way to report a bug is to enter a ticket at \myissues.  You will need to have a Google account, but they are free, so I hope that isn't too much of a hardship.  Click ``New Issue''.  If you aren't logged into your Google account, you'll be asked to log in.  This will bring up a form for you to fill out.  

There are two templates for you to choose from: ``Defect report from user'' and ``Feature request''.  In both cases, you should first enter a summary.  Then there is a text box where you can provide more information.

For a defect report, there are three questions that you should answer to help me understand the problem you are having.  They are pretty self-explanatory, so just do your best to answer them as completely as possible.  Err on the side of providing too much information rather than too little to help me figure out how to reproduce the error.  And if appropriate, please also attach the code you are using, preferably distilled down to as small a program as possible, that produces the error.

A feature request is more open-ended.  Just describe what you would like to see in a future version of TMV in as much detail as you feel is appropriate.

Another option, especially if you aren't sure if what you are seeing is really a bug or not and you want some more informal feedback, is to email the TMV discussion email list at \mygroup.  Either I or someone else on the list may be able to help you figure out the problem.

