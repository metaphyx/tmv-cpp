% !TEX root = TMV_Documentation.tex

\section{Changes from version \prevtmvversion\ to \tmvversion}
\label{Changes}

This release is mostly just some minor bug fixes and some changes in the 
installation defaults.

Here is a list of the changes from version \prevtmvversion\ to \tmvversion.  
(See \S\ref{History} for changes from previous versions to \prevtmvversion\.)
%Whenever a change is not backward compatible, meaning that code using the previous version might be broken, I mark the item with a $\times$ bullet rather than the usual $\bullet$ to indicate this.  
This release is completely backwards compatible as far as the header files and library are 
concerned.  There are a couple (minor) new features, but the library should be link-compatible
with previous versions.

\begin{itemize}

\item
Correctly detects when g++ is really clang++.  Apple has been lying about
its C++ compiler in recent MacOS systems (10.7-10.9).  They use clang++,
but they call the program g++.  This used to mess up TMV, since clang++ 
does not support OpenMP, so it would wrongly try to use OpenMP and end up 
with linking problems.  TMV now detects when clang++ is masquerading as
g++ and handles this correctly.

\item
Fixes some problems with the \texttt{install\_name} in the shared libraries.  
(Issues 3 and 7)

\item
Fixed some warnings emitted by clang++.  These weren't bugs -- just some
things that clang++ warns about that other compilers hadn't cared about.

\item
Changed the default value of \texttt{INST\_INT} to \texttt{True}, so the \tt{Matrix<int>}
templates are instantiated now unless you specifically disable them.  Enough people
had expected them to be there and asked about why they were getting 
linking errors when they used \tt{Matrix<int>}, so I decided to build them
by default now.  They don't add much to the library size.

\item
Added a default conversion from \tt{VarConjIter} to \tt{VIt}.  It was an oversight
that this wasn't possible before.  (Issue 5)

\item 
Fixed a bug in the test suite about \tt{long double} I/O.  The standard library does
not read in \tt{long double} variables at full precision.  They are only accurate to
double precision.  So I changed the I/O tests to only test \tt{long double} I/O
to double precision.  (Issue 8)

\end{itemize}
