% !TEX root = TMV_Documentation.tex

\section{Example Code}

Here are five complete programs that use the TMV library.  They are intended to showcase
some of the features of the library, but they are certainly not comprehensive.  Hopefully,
they will be useful for someone who is just getting started with the TMV library as a 
kind of tutorial of the basics.


The code includes the output as comments that start with \lstinline{//!}, so you can more easily see what is going on.
And each file listed here is also included with the TMV distribution in the \texttt{examples}
directory, so you can easily compile and run them yourself.

\subsection{Vector}

\inputcode{../examples/Vector.cpp}
%\lstinputlisting[basicstyle=\small]{../examples/Vector.cpp}
%\verbatiminput{../examples/Vector.cpp}
\vspace{12pt}

\newpage
\subsection{Matrix}

\inputcode{../examples/Matrix.cpp}
%\lstinputlisting[basicstyle=\small]{../examples/Matrix.cpp}
%\verbatiminput{../examples/Matrix.cpp}
\vspace{12pt}

\newpage
\subsection{Division}

\inputcode{../examples/Division.cpp}
%\lstinputlisting[basicstyle=\small]{../examples/Division.cpp}
%\verbatiminput{../examples/Division.cpp}
\vspace{12pt}

\newpage
\subsection{BandMatrix}

\inputcode{../examples/BandMatrix.cpp}
%\lstinputlisting[basicstyle=\small]{../examples/BandMatrix.cpp}
%\verbatiminput{../examples/BandMatrix.cpp}
\vspace{12pt}

\newpage
\subsection{SymMatrix}

\inputcode{../examples/SymMatrix.cpp}
%\lstinputlisting[basicstyle=\small]{../examples/SymMatrix.cpp}
%\verbatiminput{../examples/SymMatrix.cpp}

%\end{comment}

