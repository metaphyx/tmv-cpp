% !TEX root = TMV_Documentation.tex

\section{Vectors}
\index{Vector}
\label{Vector}

The \tt{Vector} class is our mathematical vector.  Not to be confused with
the standard template library's \tt{vector} class.  
Our \tt{Vector} class name is capitalized, while the STL \tt{vector} is not.
If this is not enough of a difference for you, and you are using both extensively in your code,
we recommend keeping the full \tt{tmv::Vector} designation for ours and \tt{std::vector}
for theirs to distinguish them.  Or you might want to \tt{typedef tmv::Vector} to some other name.

\tt{Vector} inherits from its base class \tt{GenVector} (i.e. ``generic vector'').
Most operations that do not 
modify the data are actually defined in \tt{GenVector} rather than \tt{Vector}, although
some routines are overridden in \tt{Vector} for speed reasons.  

The other classes that inherit from \tt{GenVector} are \tt{VectorView}, 
\tt{ConstVectorView} (both described in more detail below - see \S\ref{Vector_Views}), and
\tt{VectorComposite}, which is the base class for all arithmetic operations that
return a (logical) \tt{Vector}.  This means that any of these other objects can be used
any time a \tt{Vector} can be used in a non-assignable capacity.  For example,
\tt{Norm(v1+m*v2)} is completely valid, and will automatically create the necessary
temporary \tt{Vector} to store the result of the mathematical operation in the parentheses.

There is another template argument for a \tt{Vector} in addition to \tt{T} (which
represents the data type of the elements).  The second template argument
may be either \tt{tmv::CStyle} or \tt{tmv::FortranStyle}.  Or it may be omitted,
in which case \tt{CStyle} is assumed.  This argument governs how the element
access is performed.

With C-style indexing, the first element of a \tt{Vector} of length \tt{N} is 
\tt{v(0)} and the last is \tt{v(N-1)}.  Also, methods that take range arguments
use the common C convention of ``one-past-the-end'' for the last element;
so \tt{v.subVector(0,3)} returns a 3-element vector, not 4.

With Fortran-style indexing, the first element of the vector is \tt{v(1)} and the 
last is \tt{v(N)}.  And ranges indicate the first and last elements, so the same
subvector as above would now be accessed using \tt{v.subVector(1,3)} to return
the first three elements.

All views of a \tt{Vector} keep the same indexing style as the original unless you
explicitly change it with a cast.  You can cast a \tt{VectorView<T,CStyle>} as
a \tt{VectorView<T,FortranStyle>} and vice versa.  Likewise for
\tt{ConstVectorView}.  

The only thing to watch out for about the indexing is that \tt{GenVector}
and \tt{VectorComposite} do not have the extra \tt{index} 
template argument and are always
indexed using the C-style convention.  Therefore, 
if you want to index a \tt{GenVector}
using the Fortran-style convention, you would need to recast it
as an object of type
\tt{ConstVectorView<T,FortranStyle>}.  A \tt{VectorComposite} would 
not generally be indexed, but if you did want to do so using the 
Fortran-style conventions, you would need to explicitly instantiate it
as a \tt{Vector<T,FortranStyle>}.

\subsection{Constructors}
\index{Vector!Constructors}
\label{Vector_Constructors}

Here, \tt{T} is used to represent the data type of the elements of the \tt{Vector}
(e.g. \tt{double}, \tt{complex<double>}, \tt{int}, etc.) and \tt{index} is either
\tt{tmv::CStyle} or \tt{tmv::FortranStyle}.  In all of the constructors the
\tt{index} template argument may be omitted, in which case \tt{CStyle} is assumed.

\begin{itemize}

\item 
\begin{tmvcode}
tmv::Vector<T,index> v()
\end{tmvcode}
Makes a \tt{Vector} of zero size.  You would normally use the \tt{resize} function later to 
change the size to some useful value.

\item 
\begin{tmvcode}
tmv::Vector<T,index> v(size_t n)
\end{tmvcode}
Makes a \tt{Vector} of size \tt{n} with \underline{uninitialized} values.
If debugging is turned on (this is actually the default - 
turn off debugging by compiling
with \tt{-DNDEBUG} or \tt{-DTMVNDEBUG}), then the values are in fact initialized to 888.  
This should help you notice
when you have neglected to initialize the \tt{Vector} correctly.

\item
\begin{tmvcode}
tmv::Vector<T,index> v(size_t n, T x)
\end{tmvcode}
Makes a \tt{Vector} of size \tt{n} with all values equal to \tt{x}

\item
\begin{tmvcode}
tmv::Vector<T,index> v(size_t n, const T* vv)
tmv::Vector<T,index> v(const std::vector<T>& vv)
\end{tmvcode}
Make a \tt{Vector} which copies the elements of \tt{vv}.
For the first one, \tt{n} specifies the length.  The second gets
the length from \tt{vv}.

\item
\begin{tmvcode}
tmv::Vector<T,index> v = tmv::BasisVector<T,index>(size_t n, int i)
\end{tmvcode}
Makes a \tt{Vector} whose elements are all \tt{0}, except \tt{v(i) = 1}.
Note the \tt{BasisVector} also has the \tt{index} template argument to 
indicate which element is meant by \tt{v(i)}.  Again, if it is omitted,
\tt{CStyle} is assumed.

\item
\begin{tmvcode}
tmv::Vector<T,index> v1(const tmv::GenVector<T2>& v2)
v1 = v2
\end{tmvcode}
Copy the \tt{Vector v2}, which may be of any type \tt{T2} so long
as values of type \tt{T2} are convertible into type \tt{T}.
The assignment operator has the same flexibility.

\item
\begin{tmvcode}
tmv::VectorView<T,index> v = 
      tmv::VectorViewOf(T* vv, size_t n, int step = 1)
tmv::ConstVectorView<T,index> v = 
      tmv::VectorViewOf(const T* vv, size_t n, int step = 1)
\end{tmvcode}
\index{Vector!View of raw memory}
Makes a \tt{VectorView} (see \S\ref{Vector_Views} below) which refers to the exact
elements of \tt{vv}, not copying them to new storage.  The parameter \tt{n}
is the number of values to include in the view.
The optional \tt{step} parameter allows a non-unit spacing between 
successive vector elements in memory. 

\end{itemize}

\subsection{Initialization}
\index{Vector!Initialization}
\label{Vector_Initialization}

There are a number of ways to initialize the elements of a \tt{Vector}:

The most straightforward way is to simply set each element individually:
\begin{tmvcode}
for(int i=0; i<N; ++i) {
    v[i] = /* whatever */
}
\end{tmvcode}

But this is often inconvenient, especially if you want to assign a particular list of 
values that does not lend itself to being put in a \tt{for} loop.  Of course you can assign
each element one line at a time, but that can be a bit unwieldy.  So another 
way is to use a C array initializer, and then create the \tt{Vector} from these values.  For example:
\begin{tmvcode}
double var[5] = { 1.2, 3.2, -9.2, -1, 3.3 };
tmv::Vector<double> v(5,var);
\end{tmvcode}
This works, but it seems a bit inefficient to use a temporary variable.  The ideal notation
in my opinion would be to put the list initializer as a parameter in the constructor:
\begin{tmvcode}
tmv::Vector<double> v(5, 
    { 1.2, 3.2, -9.2, -1, 3.3 }); // Error - invalid
\end{tmvcode}
Unfortunately, this is not valid C++ syntax\footnote{
Apparently they are working on this shortcoming as part of the new so-called C++0x 
standards update.}.
One way to get around this is to use:
\begin{tmvcode}
tmv::Vector<double> v(5, 
    (double[]) { 1.2, 3.2, -9.2, -1, 3.3 }); // Usually OK
\end{tmvcode}
although not all compilers accept this code as valid.  I'm not even sure whether
it is ANSI-compliant C++ code or not\footnote{
The common editor, vim, doesn't seem to think so.  
It flags the braces in red, which is its indicator for a syntax error.  But most compilers
seem to compile it correctly.}.

So to address this issue, we decided to follow the syntax of \tt{std::ostream}
and use the \tt{<<} operator to initialize a \tt{Vector}.  The idea is that we are 
sending information into the vector.  

After the first 
\tt{<<}, the elements may be separated by either commas or more \tt{<<}
operators:
\begin{tmvcode}
tmv::Vector<double> v(5);
v << 1.2, 3.2, -9.2, -1, 3.3;
v << 1.2 << 3.2 << -9.2 << -1 << 3.3;
\end{tmvcode}
\index{Vector!List initialization}
There must be precisely as many values as there are elements in the \tt{Vector},
or a \tt{tmv::ReadError} will be thrown.
\index{Exceptions!ReadError}

One advantage of this method for initialization is that the values do not need to be simple numbers.
Normally the elements in a list initializer need to be numerical literals, or at least to be 
computable at compile time.
With the above syntax, the elements can be variables, 
or even values returned from functions.  Anything that is convertible
into the element type of the \tt{Vector} will work.

Also, \tt{v} can be re-initialized with this method at any time, so it is not limited to 
initialization at construction.  And finally, \tt{v} can be a \tt{VectorView}, not just a \tt{Vector}.


\subsection{Access}
\index{Vector!Access methods}
\label{Vector_Access}

\begin{itemize}
\item
\begin{tmvcode}
size_t v.size() const
\end{tmvcode}
\index{Vector!Methods!size}
Returns the size (length) of \tt{v}.

\item
\begin{tmvcode}
T v[int i] const
T v(int i) const
T v.cref(int i) const
typename Vector<T>::reference v[int i]
typename Vector<T>::reference v(int i)
typename Vector<T>::reference v.ref(int i)
\end{tmvcode}
\index{Vector!Methods!operator()}
\index{Vector!Methods!operator[]}
\index{Vector!Methods!cref}
\index{Vector!Methods!ref}
The \tt{[]} and \tt{()} forms are equivalent.  Each returns the \tt{i}-th element of \tt{v}.  
With \tt{index = CStyle}, the first 
element is \tt{v(0)}, and the last element is \tt{v(n-1)}.
With \tt{index = FortranStyle}, they are \tt{v(1)} and \tt{v(n)}.

If \tt{v} is a 
\tt{const Vector}, a \tt{ConstVectorView}, or a \tt{GenVector}, 
then the return type is just the value, not a reference.

If \tt{v} is a 
non-\tt{const Vector}, then the return type is a normal reference, \tt{T\&}.

If \tt{v} is a \tt{VectorView}, then the return type is an object, which is
an lvalue (i.e. it is assignable), but which may not be \tt{T\&}.
Specifically, it has the type \tt{typename VectorView<T>::reference}.
For a real-typed \tt{VectorView}, it is just \tt{T\&}.  But for a
complex-typed \tt{VectorView}, the return type is an object that keeps track of the
possibility of a conjugation.

The main difference between the operator forms and \tt{cref} or \tt{ref} is that the latter versions do not
check for the validity of the parameter \tt{i}, even when compiling with debugging turned on.
Also, \tt{cref} and \tt{ref} always use \tt{CStyle} indexing.


\item
\begin{tmvcode}
typename tmv::Vector<T>::iterator v.begin()
typename tmv::Vector<T>::iterator v.end()
typename tmv::Vector<T>::const_iterator v.begin() const
typename tmv::Vector<T>::const_iterator v.end() const
typename tmv::Vector<T>::reverse_iterator v.rbegin()
typename tmv::Vector<T>::reverse_iterator v.rend()
typename tmv::Vector<T>::const_reverse_iterator v.rbegin() const
typename tmv::Vector<T>::const_reverse_iterator v.rend() const
\end{tmvcode}
\index{Vector!Iterators}
\index{Vector!Methods!begin}
\index{Vector!Methods!end}
\index{Vector!Methods!rbegin}
\index{Vector!Methods!rend}
These provide iterator-style access into a \tt{Vector}, which works just like
the standard template library's iterators.  If \tt{v} is a \tt{VectorView},
the iterator types are slightly different from the \tt{Vector} iterators, 
so you should declare them as \tt{typename tmv::VectorView<T>::iterator},
etc. instead.

\end{itemize}

\subsection{Resizing}
\index{Vector!Resizing}
\label{Vector_Resize}

It is possible to change the size of a vector after it has been created using the method \tt{resize}:

\begin{tmvcode}
v.resize(size_t newsize);
\end{tmvcode}
\index{Vector!Methods!resize}

This reallocates memory for the new size with {\u uninitialized} values.  
(Or if not compiling with \tt{-DNDEBUG} or \tt{-DTMVNDEBUG},
the values are in fact set to 888.)

There are no correlates to the other constructors, so after a resize, the 
values should be set directly in some fashion.

Note that there is no automatic resizing if assigning a result that is a different size. 
For example, the code:
\begin{tmvcode}
tmv::Vector<double> v(10);
tmv::Vector<double> w(20);
// w = ...
v = 5. * w;
\end{tmvcode}
will throw a \tt{tmv::FailedAssert} exception (c.f. \S\ref{FailedAssert}) (Or if compiled with 
\tt{-DNDEBUG} or \tt{-DTMVNDEBUG} to skip TMV's asserts, then it will probably produce a segmentation fault.),
rather than resizing \tt{v} to a size of 20.  

The reason for this is that such code is more often the result of a bug, rather than intent.
So if this code is what you intend, you should write the following:
\begin{tmvcode}
v.resize(20);
v = 5. * w;
\end{tmvcode}

Also, only actual \tt{Vector}s can be resized, not the views that we will discuss in the next section.

\subsection{Views}
\index{Vector!VectorView}
\index{Vector!Views}
\label{Vector_Views}

A \tt{VectorView<T>} object refers to the elements of some other object, such as a regular
\tt{Vector<T>} or \tt{Matrix<T>},
so that altering the elements in the view alters the
corresponding elements in the original object.  A \tt{VectorView}
can have non-unit
steps between elements (for example, a view of a column of a row-major
matrix).  It can also be a conjugation of the original
elements, so that
\begin{tmvcode}
tmv::VectorView<double> cv = v.conjugate();
cv(3) = z;
\end{tmvcode}
would actually set the original element, \tt{v(3)} to \tt{conj(z)}. 

Also, we have to 
keep track of whether we are allowed to alter the original values or
just look at them.  
Since we want to be able to pass these views around, it turns out that
the usual \tt{const}-ing doesn't work the way you would want.
Thus, there are two objects that are views of a
\tt{Vector}:
\tt{ConstVectorView} and \tt{VectorView}.  
The first is only allowed to view,
not modify, the original elements.  The second is allowed to modify them.
This distinction is akin to the \tt{const\_iterator} and \tt{iterator} types in the
standard template library.

One slightly non-intuitive thing about \tt{VectorView}s is that a 
\tt{const VectorView} is still mutable.  The \tt{const} in this case
means that one cannot change the components to which the view refers.
But a \tt{VectorView} is inherently an object that can be used to 
modify the underlying data, regardless of any \tt{const} in front of it.

The following methods return views to portions of a \tt{Vector}.
If \tt{v} is either a (non-\tt{const}) \tt{Vector}
or a \tt{VectorView}, then a \tt{VectorView} is returned.
If \tt{v} is a \tt{const Vector}, a \tt{ConstVectorView}, or any other \tt{GenVector},
then a \tt{ConstVectorView} is returned.  

\begin{itemize}
\item
\begin{tmvcode}
v.subVector(int i1, int i2, int istep=1)
\end{tmvcode}
\index{Vector!Methods!subVector}
This returns a view to a subset of the original vector.
\tt{i1} is the first element in the subvector.
\tt{i2} is either ``one past the end'' (C-style) or the last element
(Fortran-style) of the subvector.
\tt{istep} is an optional step size.
Thus, if you have a \tt{Vector~v} of length 10, and you want to
multiply the first 3 elements by 2, with C-style indexing, you could write:
\begin{tmvcode}
v.subVector(0,3) *= 2.;
\end{tmvcode}
To set all the even elements to 0, you could write:
\begin{tmvcode}
v.subVector(0,10,2).setZero();
\end{tmvcode}
And then to output the last 4 elements of \tt{v}, you could write:
\begin{tmvcode}
std::cout << v.subVector(6,10);
\end{tmvcode}

For Fortran-style indexing, the same steps would be accomplished by:
\begin{tmvcode}
v.subVector(1,3) *= 2.;
v.subVector(1,9,2).setZero();
std::cout << v.subVector(7,10);
\end{tmvcode}

\item
\begin{tmvcode}
v.reverse()
\end{tmvcode}
\index{Vector!Methods!reverse}
This returns a view whose elements are the same as \tt{v},
but in the reverse order

\item
\begin{tmvcode}
v.conjugate()
\end{tmvcode}
\index{Vector!Methods!conjugate}
This returns the conjugate of a \tt{Vector} as a view, so it still points
to the same physical elements, but modifying this will set the 
actual elements in memory to the conjugate of what you set.  Likewise,
accessing an element will return the conjugate of the value in memory.

\item
\begin{tmvcode}
v.view()
\end{tmvcode}
\index{Vector!Methods!view}
Returns a view of a \tt{Vector}.  This seems at first like a silly function to have, but
if you write a function that takes a mutable \tt{Vector} argument, and you want to
be able to pass it views in addition to regular \tt{Vector}s, it is easier to write the 
function once with a \tt{VectorView} parameter.  Then you only need a second 
function with a \tt{Vector} parameter, which calls the first function
using \tt{v.view()} as the argument:
\begin{tmvcode}
double foo(const tmv::VectorView<double>& v)
{ ... [modifies v] ... }
double foo(tmv::Vector<double>& v)
{ return foo(v.view()); }
\end{tmvcode}

If you are not going to be modifying \tt{v} in the function, you only need to
write one function, and you should use the base class \tt{GenVector} 
for the argument type:
\begin{tmvcode}
double foo(const tmv::GenVector<double>& v)
{ ... [doesn't modify v] ... }
\end{tmvcode}
The arguments could then be a \tt{const Vector}, a \tt{ConstVectorView},
or even a \tt{VectorComposite}.

\item
\begin{tmvcode}
v.cView()
v.fView()
\end{tmvcode}
\index{Vector!Methods!cView}
\index{Vector!Methods!fView}
Like \tt{view()} but forces the result to have C- or Fortran-style indexing respectively.

\item
\begin{tmvcode}
v.realPart()
v.imagPart()
\end{tmvcode}
\index{Vector!Methods!realPart}
\index{Vector!Methods!imagPart}
These return views to the real and imaginary parts of a complex \tt{Vector}.
Note the return type is a real view in each case:
\begin{tmvcode}
tmv::Vector<std::complex<double> > v(10,std::complex<double>(1,4));
tmv::VectorView<double> vr = v.realPart();
tmv::VectorView<double> vi = v.imagPart();
\end{tmvcode}

\end{itemize}

\subsection{Functions of a vector }
\index{Vector!Functions of}
\label{Vector_Functions}

Functions that do not modify the \tt{Vector} are defined in 
\tt{GenVector}, and so can be used for any type derived from \tt{GenVector}:
\tt{Vector, ConstVectorView, VectorView,} or \tt{VectorComposite}.

Functions that modify the \tt{Vector} are only defined for 
\tt{Vector} and \tt{VectorView}.

Some functions are invalid if T is \tt{int} or \tt{complex<int>} because they require 
a \tt{sqrt}.  If called, these functions typically return \tt{0} if there is a return value.
And if the library is compiled with debugging on (\tt{DEBUG=true} in the scons installation),
then they will throw a \tt{FailedAssert} exception.  The descriptions for such functions
will mention if they are invalid for integer types.

\subsubsection{Non-modifying functions}

Each of the following functions can be written in two ways, either as a method or a function.
For example, the expressions:
\begin{tmvcode}
double normv = v.norm();
double normv = Norm(v);
\end{tmvcode}
are equivalent.  In either case, \tt{v} can be any \tt{GenVector}.
Also, several of the functions below have multiple equivalent names.  For example,
\tt{norm1} and \tt{sumAbsElements} are equivalent, so you can use whichever one is
clearer to you in your situation.  And just to remind you, \tt{RT} refers to the 
real type associated with \tt{T}.  So if \tt{T} is either \tt{double} or \tt{complex<double>},
\tt{RT} would be \tt{double}.

\begin{itemize}

\item
\begin{tmvcode}
RT v.norm1() const
RT Norm1(v)
RT v.sumAbsElements() const
RT SumAbsElements(v)
\end{tmvcode}
\index{Vector!Methods!norm1}
\index{Vector!Functions of!Norm1}
\index{Vector!Methods!sumAbsElements}
\index{Vector!Functions of!SumAbsElements}
The 1-norm of \tt{v}: $||v||_1 = \sum_i |v(i)|$.  

Invalid for \tt{T = complex<int>}.

\item
\begin{tmvcode}
RT v.norm2() const
RT Norm2(v)
RT v.norm() const
RT Norm(v)
\end{tmvcode}
\index{Vector!Methods!norm}
\index{Vector!Functions of!Norm}
\index{Vector!Methods!norm2}
\index{Vector!Functions of!Norm2}
The 2-norm of \tt{v}: $||v||_2 = (\sum_i |v(i)|^2)^{1/2}$.
This is the most common meaning for the norm of a vector, so we
define the \tt{norm} function to be the same as \tt{norm2}.  

Invalid for \tt{T = int} or \tt{complex<int>}.

\item
\begin{tmvcode}
RT v.normSq(const RT scale=1) const
RT NormSq(v)
\end{tmvcode}
\index{Vector!Methods!normSq}
\index{Vector!Functions of!NormSq}
The square of the 2-norm of \tt{v}: $(||v||_2)^2 = \sum_i |v(i)|^2$.
In the method version of this function, you may provide an optional scale factor,
in which case the return value is equal to NormSq(scale*v) instead, 
which can help avoid underflow or overflow problems.

\item
\begin{tmvcode}
RT v.normInf() const
RT NormInf(v)
RT v.maxAbsElement() const
RT MaxAbsElement(v)
\end{tmvcode}
\index{Vector!Methods!normInf}
\index{Vector!Functions of!NormInf}
\index{Vector!Methods!maxAbsElement}
\index{Vector!Functions of!MaxAbsElement}
The infinity-norm of \tt{v}: $||v||_\infty = \max_i |v(i)|$.  

Invalid for \tt{T = complex<int>}.

\item
\begin{tmvcode}
T v.sumElements() const
T SumElements(v)
\end{tmvcode}
\index{Vector!Methods!sumElements}
\index{Vector!Functions of!SumElements}
The sum of the elements of \tt{v} $= \sum_i v(i)$.

\item
\begin{tmvcode}
RT v.sumAbs2Elements() const
RT SumAbs2Elements(v)
\end{tmvcode}
\index{Vector!Methods!sumAbs2Elements}
\index{Vector!Functions of!SumAbs2Elements}
The sum of the ``\tt{Abs2}'' values of the elements of \tt{v}: $\sum_i |real(v(i))| + |imag(v(i))|$.
This is faster than the version using real absolute values for
complex vectors and is often just as useful, depending on your purpose.  Also this version
is valid for \tt{T = complex<int>}, while the normal version is not.

\item
\begin{tmvcode}
T v.maxElement(int* i=0) const
T MaxElement(v)
T v.minElement(int* i=0) const
T MinElement(v)
\end{tmvcode}
\index{Vector!Methods!maxElement}
\index{Vector!Functions of!MaxElement}
\index{Vector!Methods!minElement}
\index{Vector!Functions of!MinElement}
The maximum/minimum element.  For complex values, there is no way to define a 
max or min element, so just the real component of each element is used.
The \tt{i} argument is available in the method versions of these function, and it is optional.
If it is present (and not 0), then \tt{*i} is set to the 
index of the max/min element returned.

\item
\begin{tmvcode}
RT v.maxAbsElement(int* i=0) const
RT MaxAbsElement(v)
RT v.minAbsElement(int* i=0) const
RT MinAbsElement(v)
\end{tmvcode}
\index{Vector!Methods!maxAbsElement}
\index{Vector!Functions of!MaxAbsElement}
\index{Vector!Methods!minAbsElement}
\index{Vector!Functions of!MinAbsElement}
The maximum/minimum element by absolute value.  $max_i |v(i)|$ or $min_i |v(i)|$.
The \tt{i} argument is available in the method versions of these function, and it is optional.
If it is present (and not 0), then \tt{*i} is set to the 
index of the max/min element returned.  

Invalid for \tt{T = complex<int>}.

\item
\begin{tmvcode}
RT v.maxAbs2Element(int* i=0) const
RT MaxAbs2Element(v)
RT v.minAbs2Element(int* i=0) const
RT MinAbs2Element(v)
\end{tmvcode}
\index{Vector!Methods!maxAbs2Element}
\index{Vector!Functions of!MaxAbs2Element}
\index{Vector!Methods!minAbs2Element}
\index{Vector!Functions of!MinAbs2Element}
The same as the above functions, but using
$|real(v(i))| + |imag(v(i))|$ instead of the normal absolute value if \tt{T} is complex.  
This is faster than doing
the normal absolute value, and for many purposes (such as finding a suitably large value
with which to scale a vector), it is just as useful.  Also these functions are
valid for \tt{T = complex<int>}, while the normal versions are not.

\end{itemize}


\subsubsection{Modifying functions}

The following functions are methods of both \tt{Vector} and \tt{VectorView},
and they work the same way in the two cases, although there may be speed differences
between them.  
All of these are usually written on a line by themselves.  However, they do
return the (modified) \tt{Vector}, so you can string them together if you want.
For example:
\begin{tmvcode}
v.clip(1.e-10).conjugateSelf().reverseSelf();
\end{tmvcode}
\index{Vector!Functions of!Norm}
would first clip the elements at \tt{1.e-10}, then conjugate each element, then finally reverse the
order of the elements.  (This would probably not be considered 
very good programming style, however.)
Likewise, the expression:
\begin{tmvcode}
foo(v.clip(1.e-10));
\end{tmvcode}
which would first clip the elements at \tt{1.e-10},
then pass the resulting \tt{Vector} to the function \tt{foo}.  

\begin{itemize}

\item
\begin{tmvcode}
v.setZero();
\end{tmvcode}
\index{Vector!Methods!setZero}
Clear the \tt{Vector v}.  i.e. Set each element to 0.

\item
\begin{tmvcode}
v.setAllTo(T x);
\end{tmvcode}
\index{Vector!Methods!setAllTo}
Set each element to the value \tt{x}.

\item
\begin{tmvcode}
v.addToAll(T x)
\end{tmvcode}
\index{Vector!Methods!addToAll}
Add the value \tt{x} to each element.

\item
\begin{tmvcode}
v.clip(RT thresh)
\end{tmvcode}
\index{Vector!Methods!clip}
Set each element whose absolute value is less than \tt{thresh} equal to 0.
Note that \tt{thresh} should be a real value even for complex valued
\tt{Vector}s.  

Invalid for \tt{T = complex<int>}.

\item 
\begin{tmvcode}
v.conjugateSelf()
\end{tmvcode}
\index{Vector!Methods!conjugateSelf}
Change each element into its complex conjugate.  
Note the difference between this and \tt{v.conjugate()}, 
which returns a \underline{view} to a conjugated version of \tt{v} without
actually changing the underlying data.  This function, \tt{v.conjugateSelf()},
does change the underlying data.

\item
\begin{tmvcode}
v.reverseSelf()
\end{tmvcode}
\index{Vector!Methods!reverseSelf}
Reverse the order of the elements.  Note the difference between this and 
\tt{v.reverse()} which returns a \underline{view} to the elements in 
reversed order.

\item
\begin{tmvcode}
v.makeBasis(int i)
\end{tmvcode}
\index{Vector!Methods!makeBasis}
Set all elements to 0, except for \tt{v(i)} = 1.

\item
\begin{tmvcode}
v.swap(int i1, int i2)
\end{tmvcode}
\index{Vector!Methods!swap}
Swap elements \tt{v(i1)} and \tt{v(i2)}.

\item
\begin{tmvcode}
v.permute(const int* p)
v.permute(const int* p, int i1, int i2)
v.reversePermute(const int* p)
v.reversePermute(const int* p, int i1, int i2)
\end{tmvcode}
\index{Vector!Methods!permute}
\index{Vector!Methods!reversePermute}
The first one performs a series of swaps: (\tt{v(0)},\tt{v(p[0])}), (\tt{v(1)},\tt{v(p[1])}), ... 
The second starts at \tt{i1} and ends at \tt{i2-1} rather than
doing the whole range from 0 to \tt{n-1}.
The last two work the same way, but do the swaps in the opposite order.

Note: The indices listed in a permutation array (\tt{p}) always use 
the C-style convention, even if \tt{v} uses Fortran-style indexing.

These functions should not generally be needed in normal usage, since we now 
have a \tt{Permutation} class to encapsulate the most common of these kinds of 
operations.  See \S\ref{Permutation}

\item
\begin{tmvcode}
v.sort(Permutation& p, tmv::ADType ad=tmv::Ascend, 
      tmv::CompType comp=tmv::RealComp)
v.sort(tmv::ADType ad=tmv::Ascend, 
      tmv::CompType comp=tmv::RealComp)
\end{tmvcode}
\index{Vector!Methods!sort}
\index{Permutation!Vector sort}
Sorts the vector \tt{v}, optionally returning the corresponding permutation in \tt{p}.
If you do not care about the permutation, the second form is slightly more efficient.

The returned permutation is such that if you have stored the initial vector as
\tt{v\_i}, then you can reproduce the sort operation with: \tt{v = p * v\_i}.
More commonly, the permutation is used to effect the same permutation on a 
different vector or matrix.

The next parameter, \tt{ad}, determines whether the sorted \tt{Vector} 
will have its elements in ascending or descending order.  The possible values
are \tt{Ascend} and \tt{Descend}.  The default if omitted is to sort
in ascending order.

The final parameter, \tt{comp}, determines what component of the
elements to use for the sorting.  This is especially relevant if T is complex, 
since complex values are not intrinsically sortable.
The possible values are
\tt{RealComp}, \tt{AbsComp},
\tt{ImagComp}, and \tt{ArgComp}.
Only the first two make sense for non-complex vectors.  The default if
omitted is to sort the real values of the elements.  

\tt{AbsComp} and \tt{ArgComp} are invalid for \tt{T = complex<int>}.

\item
\begin{tmvcode}
Swap(v1,v2)
\end{tmvcode}
\index{Vector!Functions of!Swap}
Swap the corresponding elements of \tt{v1} and \tt{v2}.  Note that if v1 and/or v2 are
views, this does physically
swap the data elements, not just some pointers to the data.  This is the intuitive 
meaning of a statement like
\begin{tmvcode}
Swap(m.row(4),m.row(5));
\end{tmvcode}
Clearly what is desired by that is to swap the actual values, and this is what we actually do.

However, if \tt{v1} and \tt{v2} are both \tt{tmv::Vector} objects, rather than views, then the 
swap efficiently swaps the pointers to the data, and so takes $O(1)$ time, rather than
$O(N)$.

\end{itemize}

\subsection{Arithmetic}
\index{Vector!Arithmetic}
\label{Vector_Arithmetic}

\subsubsection{Operators}

All the usual operators work the way you would expect for \tt{Vector}s.  For shorthand in the 
following list, I use \tt{x} for a scalar of type \tt{T} or \tt{RT}, and
\tt{v} for a \tt{Vector}.  When there are two \tt{Vector}s
listed, they may either be both of the same type \tt{T}, or one may be of type 
\tt{T} and the other of \tt{complex<T>}.  Whenever \tt{v} is an lvalue,
if may be either a \tt{Vector} or a \tt{VectorView}.  Otherwise, it may be any \tt{GenVector}.

Also, I use the notation \tt{[+-]} to mean either \tt{+} or \tt{-}, since
the syntax is generally the same for these operators.
Likewise, I use \tt{[*/]} when their syntax is equivalent.

\begin{tmvcode}
v2 = -v1;
v2 = x * v1;
v2 = v1 [*/] x;
v3 = v1 [+-] v2;
v [*/]= x;
v2 [+-]= v1;
x = v1 * v2;
\end{tmvcode}
The last one, \tt{v1 * v2},
returns the inner product of two vectors, which is a scalar.  That is, the product
is a row vector times a column vector.  
\index{Vector!Arithmetic!inner product}

This is the only case (so far)
where the specific row or column orientation of a vector matters.  For the others
listed here, the left side and the right side are implied to be of the same orientation, but
that orientation is otherwise arbitrary.
Later, when we get to a matrix times a vector, the orientation 
of the vector will be inferred from context.

\subsubsection{Subroutines}

Each of the above equations use deferred calculation so that the sum or product is not calculated
until the storage is known.  The equations can even be a bit more complicated without 
requiring a temporary.  Here are some equations that do not require a temporary 
\tt{Vector} for the calculation:

\begin{tmvcode}
v2 = -(x1*v1 + x2*v2);
v2 += x1*(x2*(-v1));
v2 -= x1*(v1/=x2);
\end{tmvcode}

The limit to how complicated the right hand side can be without using a 
temporary is set by the 
functions that the code eventually calls to perform the calculation.  
While you shouldn't ever
need to use these directly, it may help you understand when the code will require 
temporary \tt{Vector}s.  If you do use these, note that the \tt{v} parameters
are \tt{VectorView}s, rather than \tt{Vector}s.  So you would need to 
call them with \tt{v.view()} if \tt{v} is a \tt{Vector}.

\begin{itemize}

\item
\begin{tmvcode}
MultXV(T x, const VectorView<T>& v)
\end{tmvcode}
\index{Vector!Arithmetic!MultXV function}
Performs the calculation \tt{v *= x}.

\item
\begin{tmvcode}
MultXV(T x, const GenVector<T1>& v1, const VectorView<T>& v2)
\end{tmvcode}
\index{Vector!Arithmetic!MultXV function}
Performs the calculation \tt{v2 = x*v1}.

\item
\begin{tmvcode}
AddVV(T x, const GenVector<T1>& v1, const VectorView<T>& v2)
\end{tmvcode}
\index{Vector!Arithmetic!AddVV function}
Performs the calculation \tt{v2 += x*v1}.

\item
\begin{tmvcode}
AddVV(T x1, const GenVector<T1>& v1, T x2, const GenVector<T2>& v2,
      const VectorView<T>& v3)
\end{tmvcode}
\index{Vector!Arithmetic!AddVV function}
Performs the calculation \tt{v3 = x1*v1 + x2*v2}.

\item
\begin{tmvcode}
T MultVV(const GenVector<T>& v1, const GenVector<T2>& v2)
\end{tmvcode}
\index{Vector!Arithmetic!MultVV function}
Performs the calculation \tt{v1*v2}.

\end{itemize}

More complicated arithmetic equations such as
\begin{tmvcode}
v1 += x*(v1+v2+v3) + (x*v3-v1) 
\end{tmvcode}
will require one or more temporary vectors, and so may be less efficient than 
you might like, but the code should return the correct result, no matter
how complicated the equation is.

\subsection{Input/Output}
\index{Vector!Input/output}
\label{Vector_IO}

The simplest output is the usual:
\begin{tmvcode}
os << v
\end{tmvcode}
where \tt{os} is any \tt{std::ostream}.
The output format is:
\begin{tmvcode}
n ( v(0)  v(1)  v(2)  ...  v(3) )
\end{tmvcode}
where \tt{n} is the length of the \tt{Vector}.

The same format can be read back using:
\begin{tmvcode}
tmv::Vector<T> v(n);
is >> v;
\end{tmvcode}
The \tt{Vector v} will be automatically resized to the correct size.
will be created according to whatever size the input \tt{Vector} is.
\index{Exceptions!ReadError}

Often, it is convenient to output only those values that aren't very small. 
This can be done using
\begin{tmvcode}
v.write(std::ostream& os, RT thresh)
\end{tmvcode}
\index{Vector!Methods!write}
which is equivalent to
\begin{tmvcode}
os << tmv::Vector<T>(v).clip(thresh);
\end{tmvcode}
but without requiring the temporary \tt{Vector}.

\subsection{Small vectors}
\index{SmallVector}
\label{SmallVector}

For small vectors, it is often the case that you know the size of the vector
at compile time.  Thus, we provide a class \tt{SmallVector} that takes an
additional template argument, \tt{N}, the size of the vector.

All the \tt{SmallVector} routines are included by:
\begin{tmvcode}
#include "TMV_Small.h"
\end{tmvcode}
\index{TMV\_Small.h}
\tt{SmallVector} does not inherit from the regular \tt{Vector} class, but it has 
essentially all the same methods, functions, and arithmetic operators. 

\subsubsection{Constructors}
\index{SmallVector!Constructors}
\label{SmallVector_Constructors}

The template argument \tt{N} below is an integer and
represent the size of the vector.
The template argument 
 \tt{index} may be either \tt{tmv::CStyle} or
\tt{tmv::FortranStyle} and has the same meanings as 
for a regular \tt{Vector}. The default is \tt{CStyle} if it is omitted.

\begin{itemize}
\item 
\begin{tmvcode}
tmv::SmallVector<T,N,index> v()
\end{tmvcode}
Makes a \tt{SmallVector} of size \tt{N}
with \underline{uninitialized} values.
If debugging is turned on (i.e. not turned off
with \tt{-DNDEBUG} or \tt{-DTMVNDEBUG}), then the values are in fact initialized to 888. 

\item
\begin{tmvcode}
tmv::SmallVector<T,N,index> v(T x)
\end{tmvcode}
Makes a \tt{SmallVector} with all values equal to \tt{x}.

\item
\begin{tmvcode}
tmv::SmallVector<T,N,index> v(const T* vv)
tmv::SmallVector<T,N,index> v(const std::vector<T>& vv)
\end{tmvcode}
Makes a \tt{SmallVector} with values copied from vv.

\item 
\begin{tmvcode}
tmv::SmallVector<T,N,stor,index> v1(const GenVector<T>& v2)
\end{tmvcode}
Makes a \tt{SmallVector} from a regular \tt{Vector}.

\item
\begin{tmvcode}
tmv::SmallVector<T,N,stor,index> v1(
      const SmallVector<T2,N,stor2,index2>& v2)
v1 = v2
\end{tmvcode}
Copy the \tt{SmallVector v2}, which may be of any type \tt{T2} so long
as values of type \tt{T2} are convertible into type \tt{T}.
The assignment operator has the same flexibility.

\item
\begin{tmvcode}
v << v0 , v1 , v2 , v3 ...
v << v0 << v1 << v2 << v3 ...
\end{tmvcode}
Initialize the \tt{SmallVector v} with a list of values.
\index{SmallVector!List initialization}

\end{itemize}

\subsubsection{Access}
\index{SmallVector!Access methods}
\label{SmallVector_Access}

The basic access methods are the same as for a regular \tt{Vector}.
(See \ref{Vector_Access}.)
However, since the size is known to the compiler, the inline calculation is able
to be a lot faster, often reducing to a trivial memory access.

The various view methods, like \tt{reverse} or \tt{subVector},
do not return a \tt{SmallVector}, so operations 
with the returned views will not necessarily be done inline.
However, you can copy the view back to a ``\tt{Small}'' object, which will be done 
inline, so that should be fast.

Also, if the code can determine the size from other objects in the calculation, then
it will be done inline.  e.g. if \tt{v} is a \tt{SmallVector}, then 
\tt{v2 = v + v.reverse()} will be done inline, since the first \tt{v} does have 
the compile-time knowledge about the size of the calculation.

\subsubsection{Functions}
\index{SmallVector!Functions of}
\label{SmallVector_Functions}

\tt{SmallVector} has exactly the same
function methods as the regular \tt{Vector}.  
(See \ref{Vector_Functions}.)
Likewise, the syntax of the
arithmetic is identical.  There are only a few methods that 
are not done inline.  

First, reading \tt{SmallVector} from a file 
uses the regular
\tt{Vector} I/O methods.  
Also, the \tt{sort} command for a \tt{SmallVector} just uses the
regular \tt{Vector} version.

