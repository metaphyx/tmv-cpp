% !TEX root = TMV_Documentation.tex

\section{History}
\label{History}

Here is a list of the changes from version to version.  Whenever a change 
is not backward compatible, meaning that code using the previous version
might be broken, I mark the item with a $\times$ bullet rather than the usual $\bullet$
to indicate this.  
Also, the bulleted lists are not comprehensive.  
In most cases, new versions fix minor bugs that I find in the old version.  I only 
list the more significant changes.

\begin{description}
\item[Version 0.1]
The first matrix/vector library I wrote.  It wasn't very good, really. 
It had a lot of the functionality I needed, like mixing complex/real, 
SV decomposition, LU decomposition, etc.  But it wasn't at all fast for 
large matrices.  It didn't call BLAS or LAPACK, nor were the native routines
very well optimized.  Also, while it had vector views for rows and columns, 
it didn't have matrix views for things like transpose.  Nor did it have any
delayed arithmetic evaluation.  And there were no special matrices.

I didn't actually name this one 0.1 until I had what I called version 0.3.

\item[Version 0.2]
This was also not named version 0.2 until after the fact.  It had most of 
the current interface for regular Matrix and Vector operations.  I added
Upper/Lower TriMatrix and DiagMatrix.  It also had matrix views and
matrix composites to delay arithmetic evaluation.  The main problem was that
it was still slow.  I hadn't included any BLAS calls yet.  And while the 
internal routines at least used algorithms that used unit strides whenever 
possible, they didn't do any blocking or recursion, which are key for large
matrices.

\item[Version 0.3]
Finally, I actually named this one 0.3 at the time.
The big addition here was BLAS and LAPACK calls, which 
helped me to realize how slow my internal code really was (although I hadn't
updated them to block or recursive algorithms yet).
I also added BandMatrix.

\item[Version 0.4]
The new version number here was because I needed some added functionality for a
project I was working on.  It retrospect, it really only deserves a 0.01 increment,
since the changes weren't that big.  But, oh well.

\begin{itemize}

\item
Added QR\_Downdate.  (This was the main new functionality I needed.)

\item
Improved the numerical accuracy of the QRP decomposition.

\item
Added the possibility of not storing U,V for the SVD.

\item
Greatly improved the test suite, and consequently found and corrected a few bugs.

\item
Added some arithmetic functionality that had been missing (like \tt{m += L*U}).

\end{itemize}

\item[Version 0.5]
The new symmetric matrix classes precipitated a major version number update.
I also sped up a lot of the algorithms:

\begin{itemize}

\item
Added SymMatrix, HermMatrix, and all associated functionality.

\item
Added blocked versions of most of the algorithms, so the non-LAPACK
code runs a lot faster.

\item
Allowed for loose QRP decomposition.

\item
Added divideInPlace().

\end{itemize}

\item[Version 0.51]
Some minor improvements:

\begin{itemize}

\item
Sped up some functions like matrix addition and assignment by adding the 
LinearView method.

\item
Added QR\_Update, and improved the QR\_Downdate algorithm.

\item
Blocked some more algorithms like TriMatrix multiplication/division, so non-BLAS
code runs significantly faster (but still much slower than BLAS).

\end{itemize}

\item[Version 0.52]
The first ``public'' release!  And correspondingly, the first with documentation
and a web site.  A few other people had used previous versions, but since the
only documentation was my comments in the .h files, it wasn't all that user-friendly.

\begin{itemize}

\item
Added saveDiv() and related methods for division control.  Also changed the 
default behavior from saving the decomposition to not saving it.

\item
Added in-place versions of the algorithms for $S = L^\dagger L$ and $S = L L^\dagger$.

\end{itemize}

\item[Version 0.53]
By popular demand (well, a polite request by Fritz Stabenau, at least):

\begin{itemize}

\item
Added the Fortran-style indexing.

\end{itemize}

\item[Version 0.54]
Inspired by my to-do list, which I wrote for Version 0.52, I tackled a few of the 
items on the list and addressed some issues that people had been having
with compiling:

\begin{itemize}

\item[$\times$]
Changed from a rudimentary exception scheme (with just one class - 
\tt{tmv\_exception}) to the current more full-featured exception hierarchy.
Also added \tt{auto\_ptr} usage instead of bald pointers to make the 
exception throws memory-leak safe.

\item
Sped up SymLUDiv and SymSVDiv inverses.

\item
Added the possibility of compiling a small version of the library and test suite.

\item[$\times$]
Consolidated SymLUDiv and HermLUDiv classes into just SymLUDiv, which now checks
whether the matrix is hermitian automatically.  

\item
Reduced the number of operations that make temporary matrices when multiple
objects use the same storage. 

\item
Specialized Tridiagonal $\times$ Matrix calculation.

\item
Added ElementProd and AddElementProd functions for matrices.

\item
Added CLAPACK and ACML as known versions of LAPACK.  

\end{itemize}

\item[Version 0.60]
This revision merits a first-decimal-place increment, since I added a 
few big features.  I also registered it with SourceForge, 
which is a pretty big milestone as well.

\begin{itemize}

\item
Added \tt{SmallVector} and \tt{SmallMatrix} with all accompanying algorithms.

\item 
Added \tt{SymBandMatrix} and \tt{HermBandMatrix} with all accompanying algorithms.

\item
Made arithmetic between any two special matrices compatible, so long as 
the operation is allowed given their respective shapes.  

\item[$\times$]
Changed \tt{QR\_Downdate()} to throw an exception rather than return false
when it fails.

\item
Added the GPL License to the end of this documentation, and a copyright
and GPL snippet into each file.

\item[$\times$]
Changed the -D compiler options for changing which types get instantiated.

\item
Split up the library into \texttt{libtmv.a} and \texttt{libtmv\_symband.a}. 

\end{itemize}

\item[Version 0.61]
A number of updates mostly precipitated by feature requests by me in my own
use of the library, as well as some from a few other users.
I also did a complete systematic edit of the documentation which 
precipitated some more changes to make the UI a bit more intuitive.

\begin{itemize}

\item[$\times$]
Changed the default storage for the \tt{Matrix} class to \tt{ColMajor}
rather than \tt{RowMajor}.

\item[$\times$]
Changed a lot of \tt{size\_t} parameters to \tt{int}.

\item[$\times$]
Removed \tt{U.MakeUnitDiag}.

\item
Sped up matrix multiplication for non-blas implementations, including
openmp pragmas to allow for multiple threads on machines that support them.

\item
Changed a few things which prevented Microsoft Visual C++ from compiling successfully.  
Thanks to Andy Molloy
for spearheading this effort and doing the lion's share of the work to make
the code compatible with the VC++ compiler.

\item[$\times$]
Removed the optional index parameter to the non-method versions of 
MaxElement, etc.  

\item
Added an optional \tt{scale} parameter to \tt{m.NormSq(scale)}.

\item[$\times$]
Added the explicit decomposition routines.  I also
got rid of the \tt{SVU}, \tt{SVV} and \tt{SVS} options for \tt{m.divideUsing(...)}, since the point 
of these was to do the decomposition without calculation $U$ and/or $V$.  This is now done
more intuitively with the explicit decomposition routines.
I also added the (hermitian) eigenvalue/eigenvector routines which used to
require using the division accessors in non-intuitive ways to calculate.

\item 
Fixed a couple of places where underflow and overflow could cause problems.  

\item
Updated the native TMV code for the singular value decomposition and hermitian 
eigenvalue calculation to use the divide-and-conquer algorithm.

\item
Added \tt{m.logDet()} method.  

\item[$\times$]
Changed \tt{m.svd().getS()} to return a \tt{DiagMatrix} rather than a \tt{Vector}.

\end{itemize}

\item[Version 0.62]

This release contains a number of bug fixes and corrections of mistakes in the documentation.
I also significantly revamped the \tt{SmallMatrix} class to make it faster at the expense of a few fewer features.  

\begin{itemize}

\item
Corrected an error with \tt{m.divIsSet()}.

\item
Corrected some errors in the documentation -- Thanks to Jake VanderPlas for pointing these out.

\item
Improved the behavior of the division accessors, also at the suggestion of Jake VanderPlas. (Thanks again!)  They used to require that the division
object had already been set or they would give an error. Now the accessors (like \tt{m.svd()}) set
the division object appropriately if it is not already set. 

\item
Added the \tt{ListInit} method for initializing the values of a \tt{Vector} or \tt{Matrix}.  Thanks to Paul
Sarli for this suggestion.  

\item[$\times$]
Significantly changed the \tt{SmallMatrix} class.  See \S\ref{SmallMatrix}, on the \tt{SmallMatrix} class 
in the documentation
for the full details of how it works now.  But in brief, here are the main changes:
\begin{itemize}
\item \tt{SmallMatrix} and \tt{SmallVector} no longer inherit from \tt{GenMatrix} and \tt{GenVector}.
\item This allowed us to remove all virtual functions and the corresponding \tt{vtable}.
\item Improved the arithmetic so more of the routines correctly do the calculation inline to allow the
compiler more opportunity to optimize the calculation.
\item Added inlined division and determinants.
\item Got rid of the "Small" views. 
\end{itemize}

\item
Consolidated some of the old header files.  

\item[$\times$]
Removed some of the \tt{ViewOf} commands that were not very clear and
which have other, clearer ways of doing the same thing:
\begin{itemize}
\item
\tt{d = DiagMatrixViewOf(m)} should now be written\\
\tt{d = DiagMatrixViewOf(m.diag())}.
\item
\tt{U = UpperTriMatrixViewOf(m)} should now be written\\
\tt{U = m.upperTri()}.
\item
\tt{U = UpperTriMatrixViewOf(m,UnitDiag)} should now be written\\
\tt{U = m.upperTri(UnitDiag)}.
\item
\tt{U = UpperTriMatrixViewOf(U,UnitDiag)} should now be written\\
\tt{U = U.viewAsUnitDiag()}.
\end{itemize}

\item
Tracked down the problems I had been having with the LAPACK \tt{dstegr} and \tt{sstegr} 
functions.  TMV now checks for some known problems with the LAPACK implementation
of \tt{stegr} sometimes failing, and it calls \tt{stedc} if there was a problem.

\item
Tested the code for memory bugs with Paul Nettle's mmgr.h code.  There were only a couple of 
minor memory leaks, which were fixed.

\item
Fixed a problem with the OpenMP version of the code giving segmentation faults with pgCC.

\item
Added the SCons installation method to automatically discover what BLAS and LAPACK libraries 
are installed on your computer.  

\item
Added compatibility for generic Fortran versions of BLAS and LAPACK.

\item
Added the CMake installation method.  Thanks to Andy Molloy for providing this.  

\item
Added hyperlinks to the PDF documentation. Also, I added an index, complete with hyperlinks as well, that should make it easier to find a particular topic that you might need help with.

\end{itemize}

\item[Version 0.63]
The biggest thing in this version is the new lowercase syntax for the
methods, which was in response to feedback from a number of TMV users, 
who didn't like the old syntax.  Hopefully, I'm not responding to the 
minority here, but I do agree that the new syntax conforms better
to common C++ standards.  There are also a couple of bug fixes, and I've
started using Google Code for my main website.

\begin{itemize}

\item[$\times$]
Changed the names of the methods to start with a lowercase.
The free functions all still start with a capital letter.
Both use camelCase, with underscores after 2 or 3 letter abbreviations
such as LU or SV to help them stand out (since the camel case doesn't do it).
This seems to be a more common standard in the C++ community and is more
self-consistent than the style I had been using.

\item[$\times$]
Changed the syntax for list initialization.
The old style was \\
\tt{v = tmv::ListInit, a0 , a1 , a2 , a3  ...}\\
The new style is \\
\tt{v << a0 , a1 , a2 , a3 ...}\\

\item
Improved the speed of the native (i.e. non-BLAS) matrix multiplication algorithm.

\item
Fixed a couple of bugs involving \tt{SmallMatrix}.  

\item
Disabled the cases where BlasRowMajor was used in CBlas implementations, 
since I had problems
with it on a particular BLAS version, and I'm not sure if the error
is in my code or in that BLAS implementation.
So, I removed these pathways for now until I can figure out what might
be going on with it.  Most calls were with BlasColMajor anyway, so this
doesn't affect very many calls.

\item
Fixed the return type of some methods that erroneously returned \tt{CStyle}
views when the source object was \tt{FortranStyle}.

\item
Added a way to access the version of TMV being used.
You can do this either in C++ code with the function \tt{TMV\_Version()}.
This returns a string that gives the version being used within your code.
There is also an executable, \tt{tmv-version}, that can be used to access
the TMV version that is installed at the moment.

\item
Added support for ups, which is probably completely irrelevant
to the majority of users.  But if you use ups for version control, you can
use the command \tt{scons install WITH\_UPS=true} to configure tmv in ups.

\end{itemize}

\item[Version 0.64]
This update mostly addresses problems involving underflow and overflow.  This had 
been item 13 on the to-do list (\S\ref{Bugs}), and it was bumped to top priority from
a bug report where underflow problems were leading to an infinite loop in the 
singular value decomposition code.  So I added more matrices to the test suite -- 
specifically a matrix with an extremely large dynamic range of singular values
(which reproduced the behavior of the bug report), a matrix that is nearly zero,
and a matrix that is very close to overflow.  These tests uncovered quite a few bugs 
in the code.  Some from overflow and underflow issues of course, but also some
imprecision issues, and some just plain bugs that I hadn't uncovered before.  So 
I think this version should be much more stable for a wider range of input matrices.

There are also a bunch of miscellaneous feature additions as well -- most notably the \tt{Permutation} class.

\begin{itemize}

\item
Added a some very difficult matrices to the test suite, and found and fixed quite a few 
bugs as a result involving underflow, overflow, and loss of precision.  

\item 
Fixed a bug that \tt{U(i,i)} was invalid if \tt{U} is non-const and is \tt{UnitDiag} even 
if the resulting value is not used in a mutable context.  

\item 
Added a new \tt{Permutation} class to encapsulate permutation operations.  

\item[$\times$] 
The \tt{getP()} methods from the LU and QRP divider objects now return
a \tt{Permutation} rather than a \tt{const int*}.  

\item 
Added \tt{m.sumElements()} and \tt{m.sumAbsElements()}.

\item 
Added \tt{v.minAbs2Element()}, \tt{v.maxAbs2Element()} and \tt{m.maxAbs2Element()}.

\item 
Added \tt{m.unitUpperTri()} and \tt{m.unitLowerTri()}.

\item 
Added \tt{b.subBandMatrix(i1,i2,j1,j2)} to \tt{BandMatrix} and \tt{SymBandMatrix}.

\item 
Made \tt{Swap(m1,m2)} and \tt{Swap(v1,v2)} efficient when the arguments are both
complete matrices or vectors, rather than views.

\item 
Added \tt{cView()} and \tt{fView()} methods to switch indexing styles of a matrix or vector.

\item 
Made arithmetic with \tt{SmallVector} and \tt{SmallMatrix} somewhat more flexible in that
the composite objects now (again actually) derive from the normal \tt{GenVector} and
\tt{GenMatrix} classes.  

\item 
Added a version of \tt{MatrixViewOf} with arbitrary steps.

\item 
Added the three C preprocessor definitions: \tt{TMV\_MAJOR\_VERSION}, 
\tt{TMV\_MINOR\_VERSION},
and \tt{TMV\_VERSION\_AT\_LEAST(major,minor)} to help users customize
their code depending on the version of TMV that is installed on a particular
computer.  

\item 
Fixed a bug in BLAS versions of U*M and L*M when U or L is real and
M is complex.

\item 
Fixed a bug in \tt{m.det()} when \tt{m} is a \tt{SmallMatrix<T,1,1>}.

\item 
Fixed a bug in an (apparently) uncommon pathway of \tt{Rank2Update}.

\item 
Researched my \tt{QRDowndate} algorithm to see if it is in the literature already.
The only similar algorithm I could find was in Bojanczyk and Steinhardt (1991), 
a paper which seems to have been overlooked by the matrix algorithms community.
It's not quite the same as mine in detail, but it is based on the same basic idea.
I added this reference to the discussion in the documentation about my algorithm.

\end{itemize}

\item[Version 0.65]

This update primarily fixed some problems for integer matrices.  In addition to some compiler
errors for some combinations of scons parameters, I also added the ability to calculate the 
determinant of integer matrices.  It had been documented (albeit somewhat poorly) that 
this would not work for integer matrices.  However, user Ronny Bergmann convinced me that
it would make sense to include the ability to produce accurate determinants for integer matrices.

\begin{itemize}

\item 
Made \tt{m.det()} produce accurate results if \tt{m} is a \tt{Matrix<int>} or 
\tt{Matrix<complex<int> >}.  Or in general if the underlying type is an integer type.

\item 
Added \tt{m.sumAbs2Elements()}.

\item 
Added \tt{m.addToAll(x)}.

\item 
Added \tt{m.unitUpperTri()} and \tt{m.unitLowerTri()} for symmetric/hermitian matrices.

\item 
Added the \tt{resize} methods to resize an existing vector or matrix, at the request of users Zhaoming Ma and Gary Bernstein.  

\item 
Added the ability to compile TMV as a shared library, rather than a static library 
using the \tt{SHARED=true} option with the scons installation method.

\item 
Added the ability to turn off TMV debug statements with \tt{-DTMVNDEBUG}.

\item 
Skip compiling functions that are invalid for integer values, like the various decompositions 
and such.  

\item 
Fixed some compiling errors with \tt{INST_INT=true}, including some linking errors 
for several combinations of compile options.

\item 
Fixed a couple bugs in the allocation code.

\item 
Increased the maximum number of iterations within the divide-and-conquer SVD
algorithm from 20 to 100, since I found a case that required 27 iterations to converge.

\item 
Fixed some places where nan's in the input values could lead to
infinite loops in certain algorithms. 

\item 
Fixed a place in the SV Decomposition which could (rarely) go into an infinite loop.

\item 
Made code more robust to broken BLAS distributions that incorrectly multiply the output matrix or vector by the \tt{beta} variable even if it is zero.  This is a problem if there are \tt{nan}'s present in the memory, since multiplies by zero doesn't set the value to zero in that case.  Now we always check for \tt{beta == 0} and zero out the memory if appropriate.

\end{itemize}

\item[Version 0.66] See \S\ref{Changes}.

\end{description}
