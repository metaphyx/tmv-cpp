% !TEX root = TMV_Documentation.tex

\section{Changes from Version \prevtmvversion\ to \tmvversion}
\label{Changes}

Here is a list of the changes from version \prevtmvversion\ to \tmvversion.  
%Whenever a change is not backward compatible, meaning that code using 
%the previous version might be broken, I mark the item with a $\times$ bullet rather 
%than the usual $\bullet$ to indicate this.  
Note: This version does not preserve backwards compatible in its functionality, so code that worked
with version \prevtmvversion\ may not still work.  The changes below that change functionality
are marked with an $\times$ bullet rather than the usual $\bullet$.

\begin{itemize}

\item[$\times$] 
Changed the behavior of the \tt{m << 1, 2, 3...} initialization.  Now the items in the initialization list should be in RowMajor order regardless of the storage order of the matrix \tt{m}.  This makes it much easier to understand the initialization, since the list can be arranged to look like the matrix you are setting.  e.g. the snippet:
\begin{tmvcode}
tmv::Matrix<double> m(3,5);
m << 3 ,  9 ,  1 ,  4 ,  2 ,
     7 , -3 ,  0 , -1 ,  4 ,
    -6 ,  2 , 11 ,  5 ,  6 ;
\end{tmvcode}
now initializes the matrix to have the values as shown in the rectangle.  However, the old behavior was to follow the storage order of the matrix, and since the default storage order is ColMajor, this would unintuitively initialize the matrix to
\begin{equation*}
\left(\begin{array}{ccccc}3 & 4 & -3 & 4 & 11   \\9 & 2 & 0 & -6 & 5   \\1 & 7 & -1 & 2 & 6 \end{array}\right) 
\end{equation*}
rather than the obviously intended matrix
\begin{equation*}
\left(\begin{array}{ccccc}3 & 9 & 1 & 4 & 2   \\7 & -3 & 0 & -1 & 4   \\-6 & 2 & 11 & 5 & 6 \end{array}\right) 
\end{equation*}

Now the code correctly initializes the matrix to the second one regardless of how the matrix intrinsically stores the values, which I believe is a much more intuitive behavior.

Of course, if you do want to initialize according to a ColMajor list of numbers for some reason, you can initialize \tt{m.transpose()} instead:
\begin{tmvcode}
tmv::Matrix<double> m(3,5);
m.transpose() << 3 ,  7 , -6 ,
                 9 , -3 ,  2 ,
                 1 ,  0 , 11 ,
                 4 , -1 ,  5 ,
                 2 ,  4 ,  6 ,
\end{tmvcode}

\item[$\times$] 
Deprecated the constructors from either a C-style array or a \tt{std::vector}.  The reason is similar to the reasoning above -- I wanted to decouple the internal storage order from the order of the elements in the initializing vector. 

To take the place of these constructors' functionality, I added iterators that iterate over the matrix in either RowMajor or ColMajor order, so the assignment can be done with the standard library's \tt{std::copy} function.

\begin{tmvcode}
const double rmar[15] = 
    { 3, 9, 1, 4, 2, 7, -3, 0, -1, 4, -6, 2, 11, 5, 6};
const double cmar[15] = 
    { 3, 7, -6, 9, -3, 2, 1, 0, 11, 4, -1, 5, 2, 4, 6};

// Old style constuctors: (now deprecated)
tmv::Matrix<double,tmv::RowMajor> m1(3,5,rmar);
tmv::Matrix<double,tmv::ColMajor> m2(3,5,cmar);

// New style assignment with iterators:
tmv::Matrix<double> m3(3,5); // Can be either StorageType
std::copy(rmar, rmar+15, m3.rowmajor_begin());
tmv::Matrix<double> m4(3,5);
std::copy(cmar, cmar+15, m3.colmajor_begin());
\end{tmvcode}

In addition, a \tt{BandMatrix} can be iterated in DiagMajor order using \tt{m.diagmajor_begin()}.

\item 
Added the ability to do comma initialization for \tt{DiagMatrix}, \tt{UpperTriMatrix}, \tt{LowerTriMatrix}, \tt{BandMatrix}, and their corresponding view types.  These are also always listed in RowMajor order regardless of the actual storage order of the matrix in memory.  Also, only the values in the corresponding parts of the matrices should be listed in the initialization.  e.g.
\begin{tmvcode}
tmv::DiagMatrix<double> d(4);
d << 4,
        9,
           11,
              -4 ;
tmv::UpperTriMatrix<double> u(4);
u << 4 ,  9 , -1 ,  0 ,
         18 ,  3 , -7 ,
               1 ,  5 ,
                   10 ;
tmv::LowerTriMatrix<double> l(4);
l <<  1 ,
     12 , -3 ,
      8 ,  5 , -9 ,
      3 , 20 , 14 ,  4 ;
tmv::BandMatrix<double> b(4,1,1);
b <<  7 ,  9 ,
     10 , -1 ,  5 ,
           8 , -6 , 21 ,
                2 ,  8 ;
\end{tmvcode}

I chose not to add similar initializers for symmetric or hermitian matrix types, because I think initialization of them is clearer in terms of only the upper or lower triangle part that is stored in memory.  So the way to initialize them would be something like:
\begin{tmvcode}
tmv::SymMatrix<double> s(4);
s.upperTri() << 2 ,  1 , -3 ,  7 ,
                    -8 , -2 , 12 ,
                          6 ,  1 ,
                               7 ;
tmv::SymBandMatrix<double> sb(4,1);
sb.lowerBand() <<  8 ,
                  12 ,  5 ,
                       -8 ,  2 ,
                             6 , 10 ;
\end{tmvcode}

\item[$\times$] 
Deprecated the routines that returned pointers to new \tt{BaseMatrix}es.  In general, using the \tt{BaseMatrix} class directly is not really very useful, so I don't think anyone is actually using these methods.  And in the version 0.9 TMV code in development, I've completely redesigned the back end.  It gets rid of the virtual class hierarchy completely using the CRTP idiom rather than virtual functions.  But this means that methods like \tt{newCopy()}, \tt{newTranspose()}, etc. are not really relevant anymore.  

If you were using these functions and are sorry to see them go, or if you are otherwise using the virtual functionality (e.g. with \tt{std::vector<tmv::BaseMatrix*>}), let me know what your use case was (at \mygroup) and we can try to figure out how best to incorporate your needs into the version 0.9 code.

\item 
Added a few more "ViewOf" functions to allow direct views of memory as different kinds of matrix types directly specifying the step in each direction.  These are more flexible than the versions that merely specify a StorageType to use.  For non-contiguous data, those weren't sufficient.  So now, you can view arbitrary data in memory as an \tt{UpperTriMatrixView}, \tt{LowerTriMatrixView},
\tt{DiagMatrixView}, \tt{BandMatrixView}, \tt{SymMatrixView}, or \tt{SymBandMatrixView} specifying the memory address of $m(0,0)$ and the step size in each direction.

\item 
Added support for the clang++ compiler.  This didn't require much editing, but it did complain about a couple things that had passed muster with the other compilers I had used.



\end{itemize}
