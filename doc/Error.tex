% !TEX root = TMV_Documentation.tex

\section{Errors and Exceptions}
\index{Exceptions}
\label{Exceptions}

There are two kinds of errors that the TMV library looks for.  The first are 
coding errors.  Some examples are:
\begin{itemize}
\item
Trying to access elements outside the range of a \tt{Vector} or \tt{Matrix}.
\item
Trying to add to \tt{Vector}s or \tt{Matrix}es that are different sizes.
\item
Trying to multiply a \tt{Matrix} by a \tt{Vector} where the number of columns 
in the \tt{Matrix} doesn't match the size of the \tt{Vector}.
\item
Viewing a \tt{Matrix} as a \tt{HermMatrix} when the diagonal isn't real.
\end{itemize}

I check for all of these (and similar) errors using assert statements.
If these asserts fail, it should mean that
the programmer made a mistake in the code.  (Unless I've made a 
mistake in the TMV code, of course.)

Once the code is working, you can make the code slightly faster by 
compiling with either \tt{-DNDEBUG} or \tt{-DTMVNDEBUG}.  I say slightly, since most of these checks 
are pretty innocuous.  And most of the computing time is usually in the depths
of the various algorithms, not in these $O(1)$ time checks of the dimensions and such.  
There are a few checks which take $O(N)$ time (e.g. making sure that the imaginary
part of a complex Hermitian matrix is actually 0 as it must be when initializing from values
that might not be 0's), but not many.

The other kind of error checked for by the code is 
where the data don't behave in the way
the programmer expected.  Here is a (complete) list of these errors:
\begin{itemize}
\item
A singular matrix is encountered in a division routine that cannot handle it.
\item
An input file has the wrong format.
\item
A Cholesky decomposition is attempted for a
hermitian matrix that isn't positive definite.
\end{itemize}

These errors are always checked for even if \tt{-DNDEBUG} or \tt{-DTMVNDEBUG} is used.
That's because they are not problems in the code per se, but rather
are problems with the data or files used by the code.  So they could
still happen even after the code has been thoroughly tested.

All errors in the TMV library are indicated by throwing an object of type
\tt{tmv::Error}.  If you decide to catch it, you can determine what went
wrong by printing it:
\begin{tmvcode}
catch (tmv::Error& e) {
  std::cerr << e << std::endl;
}
\end{tmvcode}
If you catch the error by value (i.e. without the \tt{\&}), it will print out
a single line description.  If you catch it by reference (as above), it may 
print out more information about the problem.

Also, \tt{tmv::Error} derives from \tt{std::exception} and overrides the \tt{what()}
method, so any program that catches these will catch \tt{tmv::Error} as well.

If you want to be more specific, there are a number of classes that derive from
\tt{Error}:

\subsection{FailedAssert}
\index{Exceptions!FailedAssert}
\label{FailedAssert}

The \tt{tmv::FailedAssert} exception indicates that one of the assert statements failed.
Since these are coding errors,
if you catch this one, you'll probably just want to print out the error and abort
the program so you can fix the bug.  In addition to printing the text of the
assert statement that failed (if you catch by reference), it will also indicate
the file and line number as normal assert macros do.  Unfortunately, it gives the 
line number in the TMV code, rather than in your own code, but hopefully seeing
which function in TMV found the problem will help you figure out which line
in your own code was incorrect.

If you believe that the assert failed due to a bug in the TMV code
rather than your own code, please post a bug report at \myissues.

\subsection{Singular}
\index{Exceptions!Singular}
\label{Singular}

The \tt{tmv::Singular} exception indicates that you tried to invert or divide by a matrix
that is (numerically) singular.  This may be useful to catch specifically,
since you may want to do something different when you encounter a singular
matrix.  Note however, that this only detects \underline{exactly} singular
matrices.  If a matrix is numerically close to singular, but no actual 
zeros are found, then no error will be thrown.  Your results will just be 
unreliable.

\subsection{ReadError}
\index{Exceptions!ReadError}
\label{ReadError}

The \tt{tmv::ReadError} exception indicates that there was some problem reading in a matrix or 
vector from an \tt{istream} input.  If you catch this by reference and write it,
it will give you a fairly specific description of what the problem was as well as
writing the part of the matrix or vector that was read in successfully.

\subsection{NonPosDef}
\index{Exceptions!NonPosDef}
\index{Cholesky Decomposition!NonPosDef exception}
\label{NonPosDef}

The \tt{tmv::NonPosDef} exception indicates that you tried to do some operation that requires
a matrix to be positive definite, and it turned out not to be positive definite.
The most common example would be performing a Cholesky decomposition on a 
hermitian matrix.  I suspect that this is the 
most useful exception to catch specifically, as opposed to just via the \tt{tmv::Error} base class.

For example, the fastest algorithm for determining whether a matrix is 
(at least numerically) positive definite is to try the Cholesky decomposition
and catch this exception.  To wit:
\begin{tmvcode}
bool IsPosDef(const tmv::GenSymMatrix<T>& m)
{
    assert(m.isherm());
    try {
        m.view().chd();
    } catch (tmv::NonPosDef) {
        return false;
    }
    return true;
}
\end{tmvcode}
Or you might want to use Cholesky for division when possible and Bunch-Kaufman otherwise:
\begin{tmvcode}
try {
    m.divideUsing(tmv::CH);
    m.setDiv();
} catch (tmv::NonPosDef) {
    m.divideUsing(tmv::LU);
    m.setDiv();
}
x = b/m;
\end{tmvcode}
Note, however, that the speed difference between the two algorithms is only about
20\% or so for typical matrices.  So if a significant fraction of your matrices are 
not positive definite, you are probably better off always using the Bunch-Kaufman (\tt{tmv::LU}) algorithm.  
Code like that given above would probably be most useful when all of your 
matrices \underline{should} be positive definite in exact arithmetic, but you 
want to guard against one failing the Cholesky decomposition due to round-off
errors.

\index{QR Decomposition!Downdate!NonPosDef exception}
It is also worth mentioning that the routine \tt{QR\_Downdate} described in \S\ref{QRDowndate}
below will also throw the exception \tt{NonPosDef} when it fails.

\subsection{Warnings}
\index{Warnings}
\label{Warnings}

There are also a few things that are not really errors, but where the user might be doing
something wrong, or where something unexpected happened, but TMV was able to handle it.
Here is a complete list of these situations:
\begin{itemize}
\item The function \tt{makeInverseATA()} was called for a matrix with more columns than rows.
The result in this case is really $(AA^\dagger)^{-1}$ instead of $(A^\dagger A)^{-1}$.
\index{Warnings!makeInverseATA}
\index{Matrix!Methods!makeInverseATA}
\index{BandMatrix!Methods!makeInverseATA}
\index{SymMatrix!Methods!makeInverseATA}
\index{SymBandMatrix!Methods!makeInverseATA}
\item The function \tt{norm2()} or \tt{condition()} was called without the SV decomposition
being previously calculated (or saved) for the matrix.  This is often inefficient, but if it is really what you
want, then \tt{doNorm2()} or \tt{doCondition()} will bypass the warning.
\index{Matrix!Methods!norm2}
\index{UpperTriMatrix!Methods!norm2}
\index{BandMatrix!Methods!norm2}
\index{SymMatrix!Methods!norm2}
\index{SymBandMatrix!Methods!norm2}
\index{Matrix!Methods!condition}
\index{UpperTriMatrix!Methods!condition}
\index{BandMatrix!Methods!condition}
\index{SymMatrix!Methods!condition}
\index{SymBandMatrix!Methods!condition}
\index{Warnings!norm2 and condition}
\item The LAPACK function \tt{dstegr} (or \tt{sstegr}) had an error, and the function
\tt{dstedc} (or \tt{sstedc}) was called instead.  The \tt{dstegr} calculation was wasted,
although I don't think it is knowable a priori that this might happen.
\index{LAPACK!dstegr/sstegr issues}
\index{Warnings!LAPACK ?stegr}
\item A LAPACK function requested more workspace than was provided.  This shouldn't happen
anymore, since I switched to using the LAPACK workspace queries.  But the code to check
for this is still there.
\index{LAPACK!Workspace issues}
\index{Warnings!LAPACK workspace}
\end{itemize}

The default way of handling these situations is to print a message to \tt{std::cout}.  This is often
appropriate, since it is probably something the programmer wants to know about.
The first two items, in fact, are probably coding errors.

However, this behavior is clearly not always acceptable.  So we provide a function to 
turn off the warnings:
\begin{tmvcode}
void tmv::NoWarnings();
\end{tmvcode}
\index{Warnings!Turn off}
\index{NoWarnings}

If you want to keep the warnings active, but don't like them being written to \tt{std::cout}, then 
you can use:
\begin{tmvcode}
std::ostream* tmv::WriteWarningsTo(std::ostream* newos)
\end{tmvcode}
\index{Warnings!Redirect to a file}
\index{WriteWarningsTo}
This lets you redirect the warnings to some other \tt{ostream}.  It could be some log file
that you look at later.  Or it could even be a \tt{stringstream} so you can examine the text
of the warnings within your code and take appropriate action.

The function \tt{WriteWarningsTo} returns a pointer to the old warning stream
in case you only want to change the warning output stream temporarily.
Also if you give \tt{WriteWarningsTo} a null pointer, rather than an actual \tt{ostream},
then this will turn off the warnings.  In fact the above \tt{NoWarnings} function is 
equivalent to \tt{(void) WriteWarningsTo(0)}.  These two features in combination
let you turn off the warnings temporarily.  For example:
\begin{tmvcode}
tmv::Matrix<double> m(5,20);
// m = ....
tmv::Matrix<double> inv_mmt(5,5);
std::ostream* tempos = tmv::WriteWarningsTo(0);
m.makeInverseATA(int_mmt); // actually calculates (m*mt)^-1
tmv::WriteWarnignsTo(tempos);
\end{tmvcode}

