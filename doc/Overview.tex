% !TEX root = TMV_Documentation.tex

\section{Overview}

First, this library is provided without any warranty of any kind.  There is no guarantee
that the code will produce accurate results for all inputs.  If someone dies because
my code gave you a wrong result, \underline{do not} blame me.

OK, that was mostly for the crazies out there.  Really, I think this is a pretty good 
product, and I've been using it for my own research extensively.  So at least 
for the routines that I use, they are probably reasonably well debugged.
I also have a test suite, which tries to be comprehensive, although
occasionally I find bugs that I hadn't thought to test for in the test suite, so 
there may still be a few lurking in there.  That means the code should definitely be considered
a beta-type release.  Hence the ``0.'' version number.  I'm also still adding functionality,
so there may be interface changes from one version to the next if I decide I 
want to do something a different way.  Just be warned.  Any such changes will be 
mentioned in \S\ref{Changes} of subsequent releases.  And see \S\ref{History}
for changes introduced in previous versions.

\subsection{Features of TMV}

The Template Matrix/Vector (TMV) Library is a C++ class library designed to make
writing code with vectors and matrices both transparent and fast.  Transparency 
means that when you look at your code months later, it is obvious what the code
does, making it easier to debug.  Fast means the execution time of the code should
be as fast as possible - this is mostly algorithm dependent, so we want the 
underlying library code to use the fastest algorithms possible.

If there were another free C++ Matrix library available that satisfied these requirements
and provided all (or even most) of the functionality I wanted, I probably would
not have written this.  But, at least when I started writing this, the available matrix libraries
were not very good.  Either they didn't have good operator overloading, or they 
didn't do complex matrices well, or they didn't include singular value decompositions,
or something.  Anyway, since I did decide to write 
my own library, hopefully other people can benefit from my efforts and will find 
this to be a useful product.

Given the above basic guidelines, the specific design features that I 
have incorporated into the code include:

\begin{enumerate}
\item
\textbf{Operator overloading}

Matrix equations look like real math equations in the code. 
For example, one can write 
\begin{tmvcode}
v2 = m * v1;
v2 += 5 * m.transpose() * v1;
m *= 3.;
v2 += m1*v1 + 3*v2 + 8*m2.transpose()*v3;
\end{tmvcode}
to perform the corresponding mathematical operations.

I also define division:
\begin{tmvcode}
x = b/A;
\end{tmvcode}
to mean solve the matrix equation $A x = b$.  If $A$ has more rows than columns,
then the solution will be a least-square solution.

\item
\textbf{Delayed evaluation}

Equations like the above 
\begin{tmvcode}
v2 = m * v1;
v2 += 5 * m.transpose() * v1;
\end{tmvcode}
do not create a temporary vector before assigning or adding the result to \tt{v2}.  This makes 
the TMV code just as fast as code which uses a functional form, such as:
\begin{tmvcode}
dgemv('N',m,n,1.,mp,ldm,v1p,s1,0.,v2p,s2);
dgemv('T',m,n,5.,mp,ldm,v1p,s1,1.,v2p,s2);
\end{tmvcode}
In fact, when installed with a BLAS library, these are the exact 
BLAS functions that the above TMV statements will call behind the scenes.
So speed is not sacrificed for the sake of code legibility.

However, a more complicated equation like
\begin{tmvcode}
v2 += m1*v1 + 3*v2 + 8*m2.transpose()*v3;
\end{tmvcode}
does not have a specialized routine, 
so it will require a couple temporary \tt{Vector}s.
Generally, if a statement performs just one operation, no temporary will be needed.  
(This includes all operations with corresponding BLAS functions along with some others
that are not included in BLAS.)
More complicated equations like this last example will give the right answer, 
but may not be quite as efficient as if you expand
the code to perform one operation per line.

\item
\textbf{Template classes}

Of course, all of the matrix and vector classes are templates, so you can have 
\begin{tmvcode}
Matrix<float>
Matrix<double>
Matrix<complex<double> >
Matrix<long double>
Matrix<MyQuadPrecisionType>
\end{tmvcode}
or whatever.

\item
\textbf{Mix complex/real}

One can multiply a real matrix by a complex
vector without having to copy the matrix to a complex one for the sake of the calculation
and deal with the concomitantly slower calculation.  
Likewise for the other arithmetic operations.  

However, it does not allow mixing of underlying data types 
(\tt{float} with \tt{double}, for example), 
with the exception of simple assignments.

\item
\textbf{Views}

Operations like \tt{m.transpose()} or \tt{v.subVector(3,8)}
return ``views'' of the underlying data rather than copying to new storage.  
This model helps delay calculations, which increases efficiency.  And the syntax
is fairly obvious.  For example:
\begin{tmvcode}
v.subVector(3,8) *= 3.;
m.row(3) += 4. * m.row(0);
m *= m.transpose();
\end{tmvcode}
modifies the underlying \tt{v} and \tt{m} in the obvious ways.

Note that in the last equation, \tt{m.transpose()} uses the same storage as
\tt{m}, which is getting overwritten in the process.  The code recognizes 
this conflict and uses temporary storage to obtain the correct result.
See ``Alias checking'' below for more about this.

\item
\textbf{C- or Fortran-style indexing}

Both C- and Fortran-style (i.e. zero-based or one-based) indexing are possible for
element access of the matrices and vectors.

With C-style indexing, all matrix and vector indexing starts with 0.  
So the upper-left element of a matrix is \tt{m(0,0)}, not \tt{m(1,1)}. 
Likewise, the lower right element of an $M \times N$ matrix is \tt{m(M-1,N-1)}.
For element ranges, such as \tt{v.subVector(0,10)}, the first number is the 
index of the first element, and the second number is ``one-past-the-end'' of 
the range.  So, this would return a 10 element vector from \tt{v(0)} to
\tt{v(9)} inclusive, not an 11 element one including \tt{v(10)}.

With Fortran-style indexing, all matrix and vector indexing starts with 1.
So the upper-left element of a matrix is \tt{m(1,1)}.  Likewise, the lower right
element of an $M \times N$ matrix is \tt{m(M,N)}.  For element ranges, such as
\tt{v.subVector(1,10)}, the first number is the index of the first element, and
the second number is the last element.  So, this would return a 10 element vector
from \tt{v(1)} to \tt{v(10)} inclusive, which represents the same actual elements 
as the C-style example above.

\item
\textbf{Special matrices}

Many applications use matrices with known sparsity or a specific structure.  
The code is able to exploit a number of these structures for increased
efficiency in both speed and storage.  So far the following special matrices are
available: diagonal, upper/lower triangle, symmetric, hermitian, banded, and 
symmetric or hermitian banded.  
Special types of banded matrices, such as 
upper and lower banded, tridiagonal, or Hessenberg, may all be declared as a 
regular \tt{BandMatrix}.  The code checks the number of sub- and super-diagonals 
and uses the right algorithm when such specialization is advantageous for a 
particular calculation.

\item
\textbf{Flexible storage}

Both row-major and column-major storage are possible as an optional
extra template parameter.
For banded matrices, there is also diagonal-major storage.
This can aid I/O, which may require a particular format to mesh with another
program.  Also, some algorithms 
are faster for one storage than than the other, so it can be worth switching storage
and checking the speed difference.

\item
\textbf{Alias checking}

Expressions such as \tt{m *= m} pose a problem for many matrix libraries, since
no matter what order you do the calculation, you will necessarily overwrite elements 
that you need for later stages of the calculation.  
The TMV code automatically recognizes the
conflict (generally known as an alias) and creates the needed temporary storage.

The code only checks the addresses of the first elements of the different objects.
So expressions such as \tt{m = m.lowerTri() * m.upperTri()} will work
correctly, even though there are three types of matrices involved, since the 
address of the upper-left corner of each matrix is the same.  (This particular
expression does not even need a temporary.  The code orders the steps of this
calculation so it can be done in place.)

However, \tt{v.subVector(5,15) += v.subVector(0,10)} will be calculated 
incorrectly, since the subvectors start at different locations, so the code
doesn't notice the aliasing.  Here, elements 5-9 will be overwritten before
they are added to the left-side vector.

Therefore, some care is still needed on the part of the user.  But this limited check is sufficient
for most applications.

\item
\textbf{BLAS}

For the combinations of types for which there are existing BLAS
routines, the code can call the optimized BLAS routines instead of its own 
code.  For other combinations (or for user defined types like 
\tt{MyQuadPrecisionType} or somesuch), 
TMV has a native implementation that it uses instead, 
using blocking and recursive techniques to make the code fairly fast on modern CPUs.

This feature can be turned off at compile time if desired 
with the option \tt{WITH_BLAS=false},
although this is not generally recommended if 
BLAS is available on your machine, and speed is important for your 
application.  This is especially true if you have a highly optimized BLAS
distribution for your machine like MKL or ACML.  However, 
the native code is usually within a factor of 2 or so of even these BLAS implementations,
so if you don't care much about optimizing the speed of your code, it's not so bad to 
use the native TMV code.
\item
\textbf{LAPACK}

TMV can also be configured to call LAPACK routines behind the scenes when possible, 
which may be faster than the native TMV code.  For types that don't have LAPACK routines, 
or if you do not enable the LAPACK calls,
the code uses blocked and/or recursive algorithms, which are similarly fast.  

For almost all algorithms, 
the TMV code is approximately as fast as LAPACK routines -
sometimes faster, since most
LAPACK distributions do not use recursive algorithms yet, which are generally
slightly faster on modern machines with good compilers. 
So if you don't want to deal with getting LAPACK
up and running, it won't generally be too bad, speedwise, 
to leave the LAPACK calls turned off.

Also, it is worth noting that many LAPACK libraries are not thread safe.
So if your main program uses multiple threads, and you aren't sure whether
your LAPACK library is thread safe, you might want to compile TMV without 
LAPACK to avoid intermittent mysterious segmentation faults from the
LAPACK library.  I believe all the native TMV code is thread safe.

\begin{comment}
\item
\textbf{Comments}

The code is highly commented, especially for the more complicated algorithms.
I have this egotistical quirk that I need to understand all of the code I write.
So none of the algorithms are just taken verbatim from LAPACK or anything like that.
I force myself to understand the algorithm (usually from Golub and van Loan,
but sometimes from a journal article or something I find on the net),
write a long comment explaining how it works, and then finally write the code
from scratch.  Then I usually compare my code to the LAPACK version, at which point
I have occasionally found ways to either speed it up or make it more accurate
(which also get commented, of course).

But the point is - if you want to understand how, say, the blocked Bunch-Kaufman
algorithm works, you'll probably do better looking at the comment in the 
TMV code than trying to decipher the LAPACK routines.  Plus, the code itself is
usually a lot more readable, since the arithmetic is written with the operator
overloads and such rather than the cryptic BLAS function names.
\end{comment}

\end{enumerate}

\subsection{Basic usage}

All of the basic TMV classes and functions, including the \tt{Vector} and \tt{Matrix} 
classes, can be accessed with
\begin{tmvcode}
#include "TMV.h"
\end{tmvcode}
\index{TMV.h}
The special matrices described below (other than diagonal and triangular matrices) 
are not included in this header file.
See their sections for the names
of the files to include to access those classes.

All of the TMV classes and functions reside in the namespace \tt{tmv}. 
And of course, they are all templates.
So if you want to declare a $10 \times 10$ \tt{Matrix}, one would write:
\begin{tmvcode}
tmv::Matrix<double> m(10,10);
\end{tmvcode}
\index{Namespace tmv}

If writing \tt{tmv::} all the time is cumbersome, one can use \tt{using}
statements near the top of the code:
\begin{tmvcode}
using tmv::Matrix;
using tmv::Vector;
\end{tmvcode}
Or, while generally considered bad coding style, one can import the whole namespace:
\begin{tmvcode}
using namespace tmv;
\end{tmvcode}
Or, you could use typedef to avoid having to write the template type as well:
\begin{tmvcode}
typedef tmv::Matrix<double> DMatrix;
typedef tmv::Vector<double> DVector;
\end{tmvcode}
In this documentation, I will usually write \tt{tmv::} with the class names to help remind the reader
that it is necessary, especially near the 
beginnings of the sections.  
But for the sake of brevity and readability, I sometimes omit it.

\subsection {Data types}

Throughout most of the documentation, I will write \tt{T} for the underlying type.
Wherever you see \tt{T}, you should put \tt{double} or \tt{std::complex<float>} or whatever.

For a user defined type, like \tt{MyQuadPrecisionType} for example, the main requirements are that 
in addition to the usual arithmetic operators, the functions:
\begin{tmvcode}
std::numeric_limits<T>::epsilon()
sqrt(T x) 
exp(T x) 
log(T x)
\end{tmvcode}
\index{User-defined types}
need to be defined appropriately, where \tt{T} is your type name.  See \S\ref{Install} for details about
compiling the library for types other than \tt{double} and \tt{float}.

Some functions in this documentation will return a real value or require a real argument, 
even if \tt{T} is complex.  In these cases, I will write \tt{RT} to indicate 
``the real type associated with \tt{T}''.  Similarly, there are places where \tt{CT}
indicates ``the complex type associated with \tt{T}''.

It is worth noting that \tt{Matrix<int>} is possible as well if you compile the library
with the SCons option \tt{INST\_INT=true}. 
However, only simple 
routines like multiplication and addition will give correct answers.  If you try to 
divide by a \tt{Matrix<int>}, for example, the required calculations are impossible 
for \tt{int}'s,
so the result will not be correct.  But since the possibility of multiplication of 
integer matrices seemed desirable, we do allow them to be used.  
\emph{Caveat programor}.  If debugging is turned on (with \tt{DEBUG=true} in the scons installation), 
then trying to do anything that requires
\tt{sqrt} or \tt{epsilon} for \tt{int}s will result in a runtime error.  Specifically a 
\tt{FailedAssert} exception will be thrown.
\index{Matrix!integer types}

The one somewhat complicated algorithm that is available for integer types is
the determinant, since the determinant of an integer matrix is always an integer.
The algorithm to do this is a so-called integer-preserving algorithm from Bareiss (1968)
that only uses division when it is known that the result will not require a fractional part.
I've extended his algorithm to \tt{complex<int>} as well, so those are also accurate.
\index{Matrix!Determinant!integer types}

\subsection{Getting the version of TMV}
\index{Version of TMV}

At times it can be useful to be able to access what version of TMV is installed on a 
particular machine, either to log the information as part of the meta-data about a
run of a program, or even to modify the code depending on which features of TMV
are available.  To address this, we provide three ways to access this information.

First, there is a function you can call from within your program:
\begin{tmvcode}
std::string TMV_Version();
\end{tmvcode}
\index{TMV\_Version}
For this release, this function returns the string ``\tttmvversion''.  This is useful for inserting
into a log file or anywhere else that you want to record the version somewhere.

Second we also provide three C preprocessor definitions:
\begin{tmvcode}
TMV_MAJOR_VERSION
TMV_MINOR_VERSION
TMV_VERSION_AT_LEAST(major,minor)
\end{tmvcode}
\index{TMV\_MAJOR\_VERSION}
\index{TMV\_MINOR\_VERSION}
\index{TMV\_VERSION\_AT\_LEAST}
\index{Conditional compilation}
The first two are defined to be \tmvmajorversion\ and \tmvminorversion\ for this release.
The third can be used with an \tt{#if} directive to change what code you want to compile
according the the version of TMV that is installed.
For example, in version 0.65 we added the \tt{m.addToAll(x)} function.  So you could
write:
\begin{tmvcode}
#if TMV_VERSION_AT_LEAST(0,65)
m.addToAll(1.0);
#else
for(int i=0;i<m.nrows();++i) 
    for(int j=0;j<m.ncols();++j)
        m(i,j) += 1.0;
#endif
\end{tmvcode}

And finally, we also include a bash script called \tt{tmv-version} that will output
the version number, so you can run this
directly from the command line.  For this release, this script produces the output:\\ \\
\texttt{\$ tmv-version} \\
\tttmvversion \\
\texttt{\$}\\ \\
This can be useful as part of a larger script if you want to log the TMV version from that
rather than from within a C++ program.  

Hopefully these three methods will satisfy any potential manner in which you might want 
to access the version of TMV that is on your system.  If there is something I missed, and 
none of these work for you, please
let me know, and I'll probably be happy to provide another mechanism in future releases.

\subsection {Notations used in this document}

There are three fonts used in this document.  First, for the main text, we use times.
Second, as you have no doubt already noticed, \tt{typewriter font} is used to 
indicate bits of code.  And finally, when I discuss the math about matrices, I 
use $italics$ -- for example, $v_2 = m * v_1$.

Also, my code syntax in this documentation is not very rigorous, aiming to maximize
readability of the code, rather than including all of the type specifications for everything.

I tend to freely mix the syntax of how a function is used with how
it is declared in order to try to provide all of the information you will need on one line.  
For example, the constructor listed as:
\begin{tmvcode}
tmv::Vector<T,A> v(int n, const T* vv)
\end{tmvcode}
is actually declared in the code as:
\begin{tmvcode}
template <class T, int A>
tmv::Vector<T,A>::Vector(int n, const T* vv);
\end{tmvcode}
and when it is used in your source code, you would write something like:
\begin{tmvcode}
int n = 5;
const double vv[n] = {1.2, 3.1, 9.1, 9.2, -3.5};
tmv::Vector<double,tmv::CStyle> v(n,vv);
\end{tmvcode}
So, the notation that I use in the documentation for this constructor is kind of a hybrid between the declaration syntax and the use syntax.  The intent is to improve readability, but
if you are ever confused about how to use a particular method, you should look at
the \tt{.h} header files themselves, since they obviously have the exactly accurate
declarations.  

